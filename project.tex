\documentclass{Assignment}
\usepackage{tikz}
\usepackage{pdfpages}
\usetikzlibrary{graphs}
\title{Graph Theory}
\begin{document}


\section{Task 1}
\subsection*{Introduction}
Main idea of this project is to understand graphs  as a mathematical set and also in a computer science perspective. Generally graphs are thought of as usually thought of as functions that are made of axis i.e x and y, and well the z-th axis. But mathematicians usually study graphs made up of set of vertices(points) with edges(curves) connecting these vertices. And these graphs have various applications in our real lives, this includes flight paths, social networks like friend groups and puzzle games. In this project our main focus would be on complete graphs.\\
\\
 I started preparing researching by starting with firstly this stage, understanding what a simple graph is, from a mathematical view point and how it can be understood from a computer science view point. From graphs we get to the Ramsey theory which is a problem relating to graphs theory and very hard to solve. Ramsey theory tries to understand how order is found in chaos. And from my understanding is Ramsey numbers are hard to find because the graphs that are generated depending on the vertices grow super-exponentially by a factor of $2^{n^2}$ this is the reason why Ramsey numbers of diagonals over 4 are not yet found.\\
  
\section*{Sub task A}
\subsection*{Graph Theory}
Graph- A graph $G = (V,E)$ is a set of objects called vertices $V$ and a pair of objects taken from $V$ called edges.\\
$V = \{v_1,v_2,v_3....\}$\\
$E = \{e_{12},e_{13}...\}$\\\\
Neighbours- Two vertices that are connected by an edge\\\\
Degree - Number of edges connected to a vertex\\\\
Directed Graph - edges have orientation\\\\
Undirected Graph -  edges have no orientation\\\\
Complete Graph $K_n$
\begin{itemize}
	\item has n vertices and all possible edges 
	\item has degrees n-1
	\item number of edges is $\frac{n(n-1)}{2}$
\end{itemize}
Adjacent Matrix\\
$$\begin{tikzpicture}
	\graph {
		1 -- {2};
		2--{3};
		3--{1};
		4--{3};

	};
\end{tikzpicture}$$
$$\begin{tabular}{c|c|c|c|c}
		&1&2&3&4\\\hline
		1&0&1&1&0\\\hline
		2&1&0&1&0\\\hline
		3&1&1&0&1\\\hline
		4&0&0&1&0\\\hline
\end{tabular}$$'
Adjacent List \\\\
\begin{center}
1 $\rightarrow$(2,3)\\
2 $\rightarrow$(1,3)\\
3 $\rightarrow$(1,2,4)\\
4 $\rightarrow$(3)\end{center}
\underline{Several applications of Graphs}\\\\
\begin{itemize}
	\item Mappings/Navigation (Uber)
	\item Social Network
	\item Puzzle games (Soduku, Chess)
\end{itemize}
\newpage
\subsection*{Ramsey Theory}
Named after mathematician Frank Ramsey, who first proved a problem in combinatorics.\\
The theory state that a monochromatic cliques in any edge labeling of large enough complete graph, or for two colours say red(r) and blue(b) be any two positive integers, the theorem states that there's a least positive integer $R(r,b)$ such that for which every blue-red edge colouring of the complete graph on $R(r,b)$ vertices contains a blue clique on r vertices or a red clique on b vertices.\\\\
\begin{itemize}
	\item In a complex and large enough system a pattern can almost be found
	\item There is order in chaos
	\item Made up of complete graphs
\end{itemize}
\subsection*{Ramsey Numbers}
Is the minimum number of vertices $n = R(r,b)$ such that all undirected simple graph of order r contains a group of order b or independent set of order n
\newpage
\section*{Sub Task B}
\begin{enumerate}
	\item n Represents the minimum number such that the complete graph $k_n$ gives for every blue-red colouring of $k_n$ a blue-coloured $k_l$ or a red-coloured $k_t$.\\\\
	When k/l or k and l are increased, we get a bigger n, this shows that n depends on k and l\\\\
	Example(The party problem)\\\\
	How many people(n) are needed to be invited to a party to guarantee that either k people know each other or l people are strangers 
	\item 
	Starting from the known Ramsey Numbers, k and l = 1, only considering the diagonal ones where k=l\\\\
	R(1,1)=1\\
	R(2,2)=2\\
	R(3,3) = 6\\
	R(4,4)=18\\
Showing that the n increases with an increase in both k and l
\end{enumerate}
\subsection*{Summary}
Firstly what i did was use the links provided to understand the theories given namely Graph Theory and Ramsey Theory. Understanding concept of graphs from a mathematical point of view together with computer science point of view. We explored applications in different fields and how they are incredibly not easy to compute. And understood that there exists order in chaos.
\newpage

\end{document}