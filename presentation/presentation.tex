\documentclass[10pt,presentation,shownotes]{myramseytheme}

\usepackage{lmodern}
\usepackage[T1]{fontenc}
\usepackage{hyperref}
\usepackage{amsmath}
\usepackage{graphicx}
\usepackage{caption}
\usepackage{tikz}
\usepackage{array}
\usepackage{pgfplots}

\usepackage{tabularx}
% Place your image in the root directory or specify the path

\pgfplotsset{compat=1.18}
\usetheme{metropolis}
\usefonttheme{professionalfonts} % Recommended for clean math fonts

% 2. CUSTOM COLORS
\definecolor{QuantumBlue}{HTML}{00CCFF} % Aqua/Cyan for highlights
\definecolor{RamseyRed}{HTML}{FF6666}   % Soft Red for Red Cliques
\definecolor{RamseyBlue}{HTML}{66FF66}  % Soft Green/Blue for Blue Cliques

% 3. APPLY CUSTOM COLORS
% Use the 'QuantumBlue' for the main highlight (e.g., section headings)
\setbeamercolor{section in head/foot}{bg=QuantumBlue, fg=black}
\setbeamercolor{subsection in head/foot}{bg=gray!50, fg=black}

% Use the custom colors for block titles
\setbeamercolor{block title}{bg=RamseyRed!30, fg=black}
\setbeamercolor{block body}{bg=white!95, fg=black}

\usepackage{lmodern} % Ensures basic Latin Modern is available
\usepackage[T1]{fontenc}
\usepackage[sfdefault]{FiraSans} % Fallback for older compilers or simpler use

\title{\texorpdfstring{Computing Ramsey numbers using a quantum computer}{Computing Ramsey numbers using a quantum computer}}

\author[M M]{Matsiri Madiba}
\date{30 October 2025}
\institute{University of Witwatersrand}
\begin{document}

	\begin{frame}
		
		\titlepage
		

	\end{frame}

\begin{frame}{Graph Theory}
	
	\begin{itemize}
		\item \textbf{A graph $G = (V, E)$:} A set of vertices ($\mathbf{V}$) connected by a set of edges ($\mathbf{E}$).
		
		
		\item \textbf{Complete Graph $K_n$:} A graph where there is an edge between pair of vertices
	
		\begin{center}

\includegraphics[scale=0.4]{complete_graph}

			\textbf{Example:} Graph with $4$ vertices and $6$ edges.
					\end{center}
	\end{itemize}
\end{frame}

\begin{frame}{Complete Graph $K_n$: Properties and Representation}
	\centering
	% Increased image scale for a larger graph visual
	\includegraphics[scale=0.4]{complete_graph} 
	
	\vspace{0.5em}
	\begin{columns}
		
		\begin{column}{0.5\textwidth} % Left Column: Properties
			\footnotesize % Smaller font for the bullet points
			\textbf{A complete graph on $n$ vertices:}
			\begin{itemize}
\item Each vertex has $n-1$ edges 
\item The total number of edges  are $M = {\frac{n(n-1)}{2}}$.
\item The total number of graphs  are $2^{M}$
			\end{itemize}
		\end{column}
		
		\begin{column}{0.6\textwidth} % Right Column: Matrix
			\centering
			\footnotesize
			\textbf{Adjacency Matrix for $K_4$:}
			\vspace{0.2em}

			$$\begin{pmatrix}
				0 & 1 & 1 & 1 \\
				1 & 0 & 1 & 1 \\
				1 & 1 & 0 & 1 \\
				1 & 1 & 1 & 0
			\end{pmatrix}$$
		\end{column}
		
	\end{columns}
\end{frame}
\begin{frame}{Graph Labeling}
\begin{itemize}
	\item Convert the graph into a binary string
	
	\item Define a consistent ordering by reading the upper triangle of the adjacency matrix.
	
	
	\begin{center}
		
		\begin{equation*}
			A =
			\begin{pmatrix}
				0 & \mathbf{x_5} & \mathbf{x_4} & \mathbf{x_3}\\
				- & 0 & \mathbf{x_2} & \mathbf{x_1}\\
				- & - & 0 & \mathbf{x_0}\\	
				- & - & - & 0
			\end{pmatrix}
		\end{equation*}
		$$\text{binary string} = (\mathbf{x_5}  \mathbf{x_4}  \mathbf{x_3}  \mathbf{x_2} \mathbf{x_1}  \mathbf{x_0} )$$
		
	\end{center}
	\item Each bit position $\mathbf{x_i}$ in the string corresponds to a unique edge
\end{itemize}



\end{frame}
 
\begin{frame}{Ramsey Theory}
	
	\begin{itemize}
		\small
		\item \textbf{Ramsey Numbers} $R(r,b)$
	
		\item \textbf{A Monochromatic Clique ($K_r$/$K_b$)}
		
	\end{itemize}
	
	\vspace{-0.5em} % Tighten space before the image
	
	\begin{center}
		\includegraphics[width=0.7\linewidth]{../graphs/order_5/graph_869}
		\\{\textbf{Example:} Graph with a 3-sized (blue) clique ($K_3$)}
	\end{center}
	
	\begin{itemize}
		\item Finding exact values for $R(r,b)$ is $\mathbf{NP}$-hard.
	\end{itemize}
	
\end{frame}
\begin{frame}{Computational Challenge}
	
	\centering % Center everything inside the frame
	\renewcommand{\arraystretch}{1.5}
	% --- Table ---
	\textbf{Total Graphs vs. Vertices}
	\footnotesize % Keep the font small for space
	
\begin{tabularx}{\textwidth}{|>{\centering\arraybackslash}X|>{\centering\arraybackslash}X|>{\arraybackslash}X|}\hline
	\textbf{Vertices ($\mathbf{n}$)} &  \textbf{Edges ($\mathbf{M}$)} & \textbf{Total Graphs $\mathbf{2^{M}}$} \\\hline
	\hline
	1 & 0 & 1 \\
	2 & 1 & 2 \\
	3 & 3 & 8 \\
	4 & 6 & 64 \\
	5 & 10 & 1,024 \\
	6 & 15 & 32,768 \\
	7 & 21 & 2,097,152 \\
	8 & 28 & 268,435,456 \\
	9 & 36 & 68,719,476,736 \\
	10 & 45 & 35,184,372,088,832 \\
	\hline
\end{tabularx}
\end{frame}
\begin{frame}
	% --- Plot ---	
	\includegraphics[scale=0.41]{my_plot}\\
	
\end{frame}
\begin{frame}{Classical Search}
	
	Example $R(3,3)$ from $n = 3$ searching for a graph with no red or blue clique
	
	\vspace{-0.5em}

	
		\begin{tabular}{|c|c|c|c|}
			\hline
			$n$&3&4&5\\\hline
			\hline
		\rotatebox{90}{counterexamples}	&	\includegraphics[scale = 0.24]{C:/Users/matsi/OneDrive/Desktop/Honours-Project/graphs/R(3,3)/counter_example_for_3_graph_1.png}&
			\includegraphics[scale = 0.24]{../graphs/order_4/k4_coloring_12}&
			\includegraphics[scale = 0.25]{C:/Users/matsi/OneDrive/Desktop/Honours-Project/graphs/R(3,3)/counter_example_for_5_graph_236.png}\\\hline
		\end{tabular}
		\begin{center}{Counterexamples of $R(3,3)$ from $n = 3, 4, 5$ }\end{center}

	All graphs on $n = 6$ vertices contain either blue or red triangle.\\
	$\implies R(3,3) = 6$\\
and the Ramsey number is $R(4,4)$ and found to be $18$

\end{frame}






\begin{frame}
	\begin{center}
		\centering
		\includegraphics[width=\linewidth]{screenshot001}
		\\{\small{Runtime growth for cliques size $b = 2-9$.}}
	\end{center}
\end{frame}
\begin{frame}{Computational Challenge}
	
	\begin{itemize}
		\item  To find  Ramsey number $R(5, 5)$ 
		
		
		\item It is bounded between $n = 43$ and $46$ vertices \cite{Exoo1993,angeltveit2024r55le46}
		% State the formula clearly and accurately
		$$\mathbf{\text{Total number of graphs} = 2^{M}}$$
		
		\item The search space grows exponentially large.
		\begin{itemize}
			\item For $n=43$, there are $\approx 10^{271}$ graphs.
			\item For $n=46$, there are $\approx 10^{311}$ graphs.
		\end{itemize}
\item Making it computationally exhaustive for classical computers.
	\end{itemize}

\end{frame}
%f they were to exhaustively search $\mathbf{K_{43}}$ to try and prove $R(5,5) \le 42$, they would be looking at a search space of $2^{903} \approx 10^{271}$, which is larger than the search space they just managed to conquer ($K_{47}$) by a factor of roughly $10^{54}$.Why $\mathbf{K_{47}}$ was possible, but $\mathbf{K_{43}}$ isn't:The computer search relies on pruning the search tree by using known properties of Ramsey graphs. As $n$ increases, the graph structure becomes denser and more constrained, which surprisingly helps the pruning algorithms.Lower Bounds are Harder to Rule Out: It's often easier to prove that a large, constrained graph MUST contain a clique (the $R(5,5) \le 46$ upper bound proof) than it is to prove a smaller graph is structurally impossible to construct (the $R(5,5) \le 42$ proof).Structural Constraints Fade: The smaller the graph, the fewer constraints the general Ramsey theory imposes on its possible structure. The search space for $K_{43}$ is simply too vast and too loosely constrained for current algorithms to rule out the non-existence of a counterexample.

 

\begin{frame}{Quantum Computing: The Qubit}
	
	\begin{itemize}
		\small
		\item \textbf{Qubit ($\mathbf{|\psi\rangle}$):} The fundamental unit of quantum information.
		
				
		\item \textbf{Basis States:}
		
		$$|0\rangle = \begin{pmatrix}
			1\\0
		\end{pmatrix} \quad \text{and} \quad |1\rangle = \begin{pmatrix}
			0\\1
		\end{pmatrix}$$
		\begin{center}

		\begin{tabular}{|c|c|}
			\hline
			$|0\rangle$  & $|1\rangle$  \\ \hline \hline
			\includegraphics[scale =0.3]{0_state} & \includegraphics[scale =0.3]{1_state} \\
			\hline
		\end{tabular}
		\end{center}

	\end{itemize}
	
\end{frame}
\begin{frame}{Quantum Computing principles}
\begin{itemize}


\item \textbf{Superposition:} 
$$|\psi\rangle = \alpha|0\rangle + \beta|1\rangle = \begin{pmatrix}
	\alpha\\
	\beta
\end{pmatrix}$$

\item \textbf{Entanglement(2>qubits)}
$$|\psi\rangle = \alpha|00\rangle + \beta|11\rangle = \begin{pmatrix}
	\alpha\\
	0\\
	0\\
	\beta
\end{pmatrix} $$ 
Where $\mathbf{|\alpha|^2 + |\beta|^2 = 1}$.\\
\item \textbf{Scalability} The solution space increases exponentially $2^Q$.
\item \textbf{Measurement} When measured superposition collapses.
%\begin{center}
%	\includegraphics[width=0.4\linewidth]{superposition}\\
%	\small{\textbf{Example:} Qubit in superposition}
%\end{center}

\item \textbf{Quantum Power:} The principles of \textbf{superposition} and \textbf{entanglement} enable quantum parallelism.
\end{itemize}
\end{frame}
% Decrease font size for the entire table content
\begin{frame}{Quantum Gates}

\begin{itemize}
	\item Qubits are manipulated and transformed using quantum gates.
	\item These are similar to Classical Logic gates.
	
	\item Examples of gates,
	\begin{itemize}
		\item \textbf{Hadamard ($\mathbf{H}$):} Creates \textbf{superposition} from basis states.
		$$ H = \frac{1}{\sqrt{2}}\begin{pmatrix} 1 & 1 \\ 1 & -1 \end{pmatrix} \quad \implies \quad H|0\rangle = \frac{1}{\sqrt{2}}\left(|0\rangle+|1\rangle\right) $$
		
		\item \textbf{Pauli Gates ($\mathbf{X, Y, Z}$):} Perform rotations around the Bloch sphere axes.
		\item \textbf{C-NOT} Gate - flips the \textbf{target} qubit if the \textbf{control} qubit is in state $|1\rangle$
	\end{itemize}
	
	
\end{itemize}




\end{frame}

\begin{frame}{Quantum Circuits}
	
	\begin{itemize}
		\item Quantum circuits is a sequence of gates applied on a qubit(s)
		\item Computation flows from left to right
		\item  Gates ($H$ and $CNOT$) are applied to get superposition and entanglement
	\end{itemize}
 
	\includegraphics[scale=3.0]{../plots/quantum_circuit}\\
	\centering
		{\small{A basic quantum circuit showing  gates}}


\end{frame}
\begin{frame}{Quantum computers limitations}
	\begin{itemize}
		\item Decoherence(Noise)- Limits number of gates applied
		\item Number of Qubits - Few number
		
	\end{itemize}
\end{frame}

\begin{frame}{Mapping Ramsey Number to Quantum Computing}
	
	\begin{itemize}
		\item The \textbf{Hamiltonian ($\mathcal{{H}}$)} represents the total energy of a system.
		$$\mathcal{{H}} = T + V$$

		\item Quantum computers can be used for the system's ground state.
		
		$$\mathcal{{\hat{H}}}|\psi\rangle = E|\psi\rangle$$

		
		\item Ramsey numbers can be encoded as a Hamiltonian
		
		\item The defining the Hamiltonian as $$\mathcal{{H}}_c(a) = C^n_r(a)+C^n_b(a)$$
		$C^n_r(a) $ total count of red cliques\\ $C^n_b(a)$ total count of blue cliques \\
		\small
		\item When the Hamiltonian acts on the state we get $\mathcal{{\hat{H}}}|\psi\rangle = E|\psi\rangle$
	\end{itemize}
	
\end{frame}
\begin{frame}{Mapping Ramsey Number to Quantum Computing}
	
	\begin{itemize}
	
		\item The goal is to find the graph $\mathbf{a}$ that minimises this energy,
		\begin{align*}
		& \mathcal{{H}}_c(\mathbf{a}) = C^n_r(\mathbf{a})+C^n_b(\mathbf{a})\\
		&\mathcal{{\hat{H}}}|\mathbf{a}\rangle = E_{min}|\mathbf{a}\rangle
		\end{align*}

		
		\item If $E_{\min}= 0$  a counterexample exists, $R(r,b)>n$ 
		\item But when $E_{\min}> 0$ then $R(r, b) = n$.
	\end{itemize}

\end{frame}

\begin{frame}{Mapping Ramsey Number to Quantum Computing}
\begin{itemize}
	\item Each graph is represented as a state on a quantum computer,
	\item The edges on a graph are mapped to qubits,
	\item These exists as a superposition ,
	\[|\psi\rangle = \frac{1}{\sqrt{2^{M}}}\left(|0\rangle +|1\rangle+|2\rangle+...|2^{M}-1\rangle\right)\]
	in vector notation
	\[=\frac{1}{\sqrt{2^{M}}}\begin{pmatrix}
x_1 &x_2& x_3&...&x_{M-1}
	\end{pmatrix}\]
\end{itemize}
	
\end{frame}
\begin{frame}{Quantum Algorithms}
	\begin{itemize}
		\item Grover's Algorithm-$O(\sqrt{N})$ (Quadratic Speedup)
		\item The Quantum Approximate Optimization Algorithm (QAOA)
		\item Variational Quantum Eigensolver 
	\end{itemize}
\end{frame}
\begin{frame}{The Quantum Approximate Optimization Algorithm (QAOA)}
	
	\begin{itemize}
		\item A variational algorithm designed to solve optimization problems.

		\item QAOA is a hybrid algorithm both classical and quantum computing:
		\begin{enumerate}
			\item Quantum : Runs a parameterized circuit to explore the solution space.
			
			\item Classical: minimise parameters
		\item  steers the system toward the ground state.
				
		\[U_c = \exp{(-i \alpha \mathcal{{H}}_c )}\]
		\[U_M = \exp{(-i \gamma \mathcal{{H}}_M )}\]
		\end{enumerate}
		
		
		\item  To find the graph that has the lowest number of cliques.
	\end{itemize}
	
\end{frame}
\begin{frame}{Mapping Ramsey Number to Quantum Computing}
	Example For $R(3,3)$ starting with $n = 3$ vertices
	\begin{center}
		\begin{tabular}{|c|c|c|c|}\hline
			\hline
			\includegraphics[scale =0.17]{../graphs/order_3/k3_coloring_0}&
			\includegraphics[scale =0.17]{../graphs/order_3/k3_coloring_1}&
			\includegraphics[scale =0.17]{../graphs/order_3/k3_coloring_2}&
			\includegraphics[scale =0.17]{../graphs/order_3/k3_coloring_3}\\\hline
			\hline
			\includegraphics[scale =0.17]{../graphs/order_3/k3_coloring_4}&
			\includegraphics[scale =0.17]{../graphs/order_3/k3_coloring_5}&
			\includegraphics[scale =0.17]{../graphs/order_3/k3_coloring_6}&
			\includegraphics[scale =0.17]{../graphs/order_3/k3_coloring_7}\\\hline
		\end{tabular}
	\end{center}
\end{frame}
\begin{frame}{Mapping Ramsey Number to Quantum Computing}
	\begin{center}
		\includegraphics[scale = 0.4]{../QAOA/energy_hlines_distribution_3}
	\end{center}
	
\end{frame}
\begin{frame}

		\includegraphics[scale=0.4]{../QAOA/prob_distribution_3.png}
		
\end{frame}
\begin{frame}
	\begin{center}
				\centering Results from real quantum hardware (Qiskit ibm\_kingston)
	\includegraphics[scale=0.4]{../QAOA/quantum_measurement_binary_distribution_v2}
	\end{center}
Total Shots = 10000, Program = sampler, Quantum Computer= ibm\_kingston, classical optimizer = COBYLA, 3 QAOA reps
\end{frame}

\begin{frame}
\begin{center}
	\includegraphics[scale = 0.45]{../QAOA/energy_hlines_distribution_6}
\end{center}

\end{frame}
\begin{frame}
	
	\includegraphics[scale=0.32]{../QAOA/prob_distribution_6.png}
	With the following graph numbers 24851 and 29271 having 2 cliques each
\end{frame}

\begin{frame}{References}
	\begin{thebibliography}{9}
 \bibitem{burr1981generalized}
 S.A. Burr, Generalized Ramsey theory for graphs, \textit{Discrete Mathematics}, 1981.
		\bibitem{Exoo1993}
		G. Exoo, A lower bound for \( R(5,5) \), \textit{J. Graph Theory}, 1993.
		\bibitem{angeltveit2024r55le46}
		V. Angeltveit et al., \( R(5,5) \leq 46 \), arXiv:2401.08077, 2024.
		
		\bibitem{PhysRevA.93.032301}
		F. Gaitan and L. Clark, Ramsey numbers and adiabatic quantum computing, \textit{Phys. Rev. A}, 2016.
	\end{thebibliography}
\end{frame}
\begin{frame}
	% --- Plot ---	
	\includegraphics[scale=0.41]{my_plot}
	
\end{frame}
\begin{frame}{Quantum Gates}
	\centering
	{Visualizing Gate Effects on the Bloch Sphere}
	\begin{tabular}{|c|c|c|}
		\hline
		
		&	Initial state&Final state  \\\hline
		\hline
		Apply $H$ gate&	\includegraphics[scale =0.2]{../bloch_plots/initial}&
		\includegraphics[scale = 0.2]{superposition}\\
		\hline
		Apply $X$ gate&	\includegraphics[scale =0.2]{../bloch_plots/initial}&
		\includegraphics[scale = 0.2]{../bloch_plots/final}\\ \hline
	\end{tabular}
\end{frame}
\begin{frame}
Lower bound (\(R(5,5)\ge 43\)): In 1989, Geoffrey Exoo constructed a graph with 42 vertices that does not contain a clique of size 5 or an independent set of size 5. This proves that \(R(5,5)\) must be at least 43.
\end{frame}
\begin{frame}
	\begin{center}
		\includegraphics[scale = 0.4]{../QAOA/energy_hlines_distribution_4}
	\end{center}
\end{frame}
\begin{frame}
	Results from real quantum hardware (Qiskit ibm\_kingston)	
	\includegraphics[scale=0.5]{../QAOA/prob_distribution_4.png}
	
\end{frame}
\begin{frame}
	
	\centering
	\includegraphics[width=0.7\linewidth]{../graphs/order_4/k4_coloring_33}\\
	\small{Graph with Most probability of being measured}
	\label{fig:k4coloring33}
	
\end{frame}
\end{document}
Upper bound (\(R(5,5)\le 46\)): In a 2024 paper, Vigleik Angeltveit and Brendan McKay established that \(R(5,5)\) is at most 46. This was the latest improvement on previous upper bounds of 48 and 49.


a2m​=(−1)m22m(m!)2λ2m​a0​.​