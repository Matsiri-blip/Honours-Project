\documentclass{beamer}
\usepackage[utf8]{inputenc}
\usepackage{lmodern}
\usepackage[T1]{fontenc}
\usepackage{hyperref}
\usepackage{amsmath}
\usepackage{graphicx}
\usepackage{tikz}
\usetikzlibrary{positioning, overlay-beamer-styles}
\usepackage{pgfplots}
\usepackage{lmodern}
\pgfplotsset{compat=1.18}
\usetheme{Montpellier} % Modern, clean theme
\institute{Wits University}
\title{\texorpdfstring{Computing $R(5,5)$ using a quantum computer}{Computing R(5,5) using a quantum computer}}
\author[A. Name]{Matsiri Madiba}
\date{30 May 2025}
\begin{document}
	\begin{frame}
		\titlepage
	\end{frame}
\begin{frame}{Outline}
	\begin{itemize}
		\item Introduction
		\item Background: Ramsey Theory and Quantum computing
		\item Research Objectives
		\item Methodology and Work Plan
	\end{itemize}
\end{frame}

\begin{frame}{Introduction}
	\frametitle{Introduction}
\begin{itemize}
			\item Ramsey Theory
	\begin{itemize}
		\item Graph Theory: Helps in understanding Ramsey Numbers \pause
		\item Ramsey Number \( R(5,5) \) \pause
	\end{itemize}
	\item Quantum Computing: Methods for Optimization \pause
	\item Can quantum computers solve \(R(5,5)\) \pause



	\end{itemize}
\end{frame}	
\begin{frame}{Graph Theory Basics}
	\begin{itemize}
		\item A graph \( G = (V, E) \):
		\begin{itemize}
			\item \( V \): Vertex set 
			\item \( E \): Edge set 
		\end{itemize}
	\end{itemize}
	 Representation of a graph.\\
	 
	Adjacency matrix,
		$$A_{ij}=\begin{cases} 
			1 & \text{if } e_{ij} \in E \\ 
			0 & \text{otherwise}
		\end{cases}$$
		$e_{ij}$ is an element in the edge set $E$
	%	\item Example: Complete graph \( K_3 \) (3 vertices, all edges present)
	%	\item Relevance: Ramsey numbers involve 2-edge-colorings of \( K_n \), seeking monochromatic cliques.
	
\end{frame}
\begin{frame}{Graph Theory Basics}
Types of graphs
\begin{enumerate}
	\item Complete graph
		\begin{center}
		\includegraphics[scale=0.2]{complete_graph.png}
	\end{center}\pause
	\item Empty graph
		\begin{center}
		\includegraphics[scale=0.2]{empty_graph.png}
	\end{center}
\end{enumerate}

\end{frame}
\begin{frame}{Graph Theory Basics}
 Complete graph (n-vertices)
\begin{itemize}
	\item Each vertex has $n-1$ edges
	\item There are n vertices
	\item The complete graph has of n vertices has ${\frac{n(n-1)}{2}}$	possible 2-edge-colourings,

\end{itemize}
Ramsey Theory considers complete graphs
\end{frame}
\begin{frame}{Ramsey Theory}
Ramsey Numbers \( R(r,b) \) 
\begin{itemize}
	\item clique
	\item independent set 
\end{itemize}

$$\includegraphics[scale =0.3]{clique.png}$$
\begin{center}\textit{Complete graph of 6 vertices with a 3 vertex clique}\end{center}
\end{frame}
\begin{frame}
	\begin{itemize}
\item A 2-colouring n-vertex has $2^{\left(\frac{n^2-n}{2}\right)}$ graphs.
\end{itemize}
$$\includegraphics[scale = 0.5]{plot.png}$$
\begin{center}
\small{Plot showing the growth of number of graphs with increasing n}
\end{center}
\end{frame}

\begin{frame}{Research Problem}
	\begin{itemize}
		\item Find Ramsey Number \( R(r,b) \), $r=b=5$
		\item \( R(5,5) \) has bounds \( 43 \leq R(5,5) \leq 46 \) \cite{Exoo1993, angeltveit2024r55le46}.
		\item Computational Challenge:
		\begin{itemize}
			\item Exponential colourings: \( 2^{903} \approx 10^{271} \) for \( n = 43 \).
			\item NP-hard \cite{burr1981generalized}.
		\end{itemize}
	\end{itemize}
\end{frame}
\begin{frame}{Quantum computing basics}
	\begin{itemize}
		\item Made up of Qubits\pause
		\item Unlike classical computers a qubit can be in two states \pause
		\item Leverages concepts like \underline{superposition} \pause and \underline{entanglement}
		$$	\includegraphics[width=0.5\linewidth]{Qubit}$$

		
	\end{itemize}
\end{frame}
\begin{frame}{Quantum computing basics}
	\begin{itemize}
		\item Quantum gates are used to make calculations.
		\item These gates can be represented as \underline{unitary} matrices.
	\end{itemize}
	\textbf{Examples of Quantum gates}
	\begin{itemize}
\item Hadamard gate
\item Pauli matrices
\item CNOT gate
	\end{itemize}
\end{frame}
\begin{frame}{Quantum computing}
	\textbf{Why Quantum computing?}
	\begin{itemize}
		\item Quantum computers can try many solutions faster.
		\item Classical computers would take too long to go through $ 2^{903}$ graphs.

	\end{itemize}
\end{frame}
\begin{frame}{Objectives}
	\begin{itemize}
		\item Determine \( R(5,5) \)
		\item Develop and test quantum algorithms.
		\item Evaluate quantum computing’s efficiency.
\end{itemize}
\end{frame}

\begin{frame}{Methodology}
	\begin{itemize}
		\item \textbf{Classical Methods}:
		\begin{itemize}
			\item Use binary encoding to enumerate 2-edge-colourings.
		\end{itemize}
		\item \textbf{Quantum Methods}:
		\begin{itemize}
			\item Quantum computing: Map \( R(5,5) \) into an optimization problem.
		\end{itemize}
		\item Test on small graphs first.
	\end{itemize}
\end{frame}
\begin{frame}{Work Plan}
	\begin{table}[h]

		\begin{tabular}{|p{2cm}|p{2cm}|p{5cm}|}
			\hline
			\textbf{Phase} & \textbf{Time} & \textbf{Tasks} \\
			\hline
			Phase I & Feb--May & \small{Study Ramsey theory, graph theory and quantum computing.} \\
			\hline
			Phase II & Jun--Jul & Develop classical algorithms. \\
			\hline
			Phase III & Aug--Sep & Implement quantum algorithms, test on small graphs. \\
			\hline
			Phase IV & Oct & Analyze results, put into final report. \\
			\hline
		\end{tabular}
	\end{table}
\end{frame}
\begin{frame}
	\textbf{Thank You.}
\end{frame}
\begin{frame}{References}
	\begin{thebibliography}{9}
		\bibitem{BondyMurty2008}
		J.A. Bondy and U.S.R. Murty, \textit{Graph Theory}, Springer, 2008.
		\bibitem{Exoo1993}
		G. Exoo, A lower bound for \( R(5,5) \), \textit{J. Graph Theory}, 1993.
		\bibitem{angeltveit2024r55le46}
		V. Angeltveit et al., \( R(5,5) \leq 46 \), arXiv:2401.08077, 2024.
		\bibitem{burr1981generalized}
		S.A. Burr, Generalized Ramsey theory for graphs, \textit{Discrete Mathematics}, 1981.
		\bibitem{PhysRevA.93.032301}
		F. Gaitan and L. Clark, Ramsey numbers and adiabatic quantum computing, \textit{Phys. Rev. A}, 2016.
	\end{thebibliography}
\end{frame}
\begin{frame}
$$\includegraphics[scale=0.6]{figure_1.png}$$
\end{frame}
\begin{frame}

\end{frame}
\end{document}




 Example: \( R(3,3) = 6 \)
\begin{center}
	\begin{tikzpicture}
		% Vertices of K_6
		\foreach \i in {0,...,5}
		\node[circle, draw] (v\i) at (60*\i:1) {};
		% All edges in black (neutral)
		\foreach \i in {0,...,4}
		\foreach \j in {\i,...,5}
		\ifnum \j>\i
		\draw (v\i) -- (v\j);
		\fi
		% Highlight a clique of size 3 (v0, v1, v2) in red
		\draw[red, thick] (v0) -- (v1) -- (v2) -- (v0);
	\end{tikzpicture} \\
	\small{\( K_6 \): Any 2-edge-coloring has a monochromatic triangle (clique of size 3).}
\end{center}
\end{itemize