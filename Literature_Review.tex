% ------------------------------------------------------------
\documentclass[12pt, a4paper]{Assignment}
% ------------------------------------------------------------


% ------------------------------------------------------------
% Formatting
% ------------------------------------------------------------
\usepackage{color}
\usepackage{fullpage}
% ------------------------------------------------------------
\usepackage{float}
\usepackage{graphicx} % Required for inserting images
\usepackage{amssymb}
% ------------------------------------------------------------
% Bibliography
% ------------------------------------------------------------
\usepackage{doi}
\usepackage{hyperref}
\usepackage{usnomencl}
\usepackage[square,sort&compress,comma,numbers]{natbib}
\bibliographystyle{unsrtnat}
\hypersetup{
  colorlinks,
  citecolor=blue,
  linkcolor=red,
  urlcolor=blue}
\usepackage{bibentry}
\usepackage{tikz}
\usepackage{pdfpages}
% ------------------------------------------------------------


% ------------------------------------------------------------

\author{Matsiri Madiba}
\date{April 2025}
\title{Literature Review}
% ------------------------------------------------------------


% ------------------------------------------------------------
\begin{document}
\maketitle
\section*{Comments}

From reading the paper \cite{burr1981generalized}, here I see they speak on the computational complexity of Ramsey Numbers. 
I have a little trouble in understanding Ramsey numbers as a class of $\sum_{2}^{p}$. 
I also do not get the notation or how Ramsey numbers are part of this class.
The mathematical notation in this paper is the most confusing, like what $W_{1,n}$ n-spoked wheel represent.
\\
\\
\cite{Deutsch1989}
A complete set C of compatible observables for $\mathcal{P}$ is chosen, including an input 
observable $\hat{I}$ and an observable $\hat{h}$ that is independent of $\hat{I}$ and has spectrum \{0,1\}, the halt flag. 
This part in the paper is confusing.
A halt flag is introduced not sure of its importance but im confused by this.
I do not understand how from this set what compatible observables are chosen.
\\
\\
I notice I am very bad at getting the main ideas from papers.
And even the main aims of the papers.
\section*{Introduction}
Ramsey theory only considers complete graphs. 
In the paper \cite{burr1981generalized} Burr tries to understand how hard evaluating Ramsey Numbers is.
By introducing different complexities of problems.
The paper \cite{PhysRevA.93.032301} they try getting around this by using Quantum Computing.
\newline
\\
Deutsch\cite{Deutsch1989} builds on universal quantum computer by \cite{deutsch1985quantum} from classical computing to quantum mechanics. 
We understand about the development of quantum computation theory.
He then relates universal quantum computer to the classical computer using the paper \cite{Turing1936}.
While showing quantum computer can perform powerful computations using superposition and entanglement.
\cite{rietsche2022quantum} speaks about the physical implementations of quantum computers. 
The concept of quantum computing developed in the early 1980s from papers \cite{feynman1982simulating}. 	
In the papers \cite{PhysRevA.93.032301}, \cite{doi:10.1137/S0097539796298637} and \cite{UQS} developments and applications are discussed.
\newline
\\
Quantum algorithm is used to calculate Ramsey Numbers, proposed in \cite{gaitan2012ramsey}.
A Hamiltonian is used, this was also referenced in the paper \cite{doi:10.1137/S0097539796298637}.
Wang turned the problem into a combinatorial optimization problem.
Adiabatic Quantum Optimization considering \cite{farhi2000quantum} was used.
\newline
\\
This article\cite{doi:10.1137/S0097539796298637} speaks about quantum machines from the paper \cite{Deutsch1989}.
Simon gives reasons why quantum computers have advantage over classical ones.
The focus is on quantum computing algorithm that can show how fast these computers are	.
Using the article \cite{shor1994algorithms} he explores the development of quantum polynomial-time algorithms.
\newline
\\
\cite{UQS} introduces the concept of universal quantum simulators. 
Which extends on the idea introduced by Feynman of a computer that can simulate quantum systems from the paper \cite{feynman1982simulating}. 
He also speaks on quantum gates, which was also referenced in the paper \cite{Deutsch1989}.
\newline
\\
In the first section I will explain graph theory.
And I will be expanding on this by Ramsey theory.
Extending Ramsey theory by explaining Ramsey numbers and their complexities.
\\
The second section will deal with computations and computation complexity.
Explaining why Quantum computers have more computational power over classical ones.
And how they work.
\\
And lastly explain how Quantum computers can help in solving Ramsey numbers.
\section{Graph Theory}
A study of graphs where it aims to understand relationships between points. 
A graph is set of ordered triple of objects $G = \{V(G),E(G), \psi_G\}$ called vertices $V(G)$, a pair of objects taken from $E(G)$ called edges and $\psi_G$ a function that connects the edges with vertices of G.
And $V(G)$ = \{$v_i$\} and $E(G)=\{e_j\}$ where $i$ would represent the number of vertices and $j$ would be the number of edges found in the graph.
If $e_1$ represents and edge and $v_1 ,v_2$ are vertices such that $\psi_G(e_1)=v_1v_2$, $e_1$ joins the vertices $v_1,v_2$.\cite{BondyMurty2008} 
Vertices can be represented as points and edges are represented as lines joining these points/nodes.
A graph is called finite if its edges and vertices are also finite.
A representation of graph is shown below,
\begin{figure}[H]
	\centering
	\includegraphics{random_graphs/random_graph1.png}
	\caption{Graph with 4 vertices and 3 edges}
	\label{random_graph1}
\end{figure}

An order of a graph can be represented by the number of vertices in the graph.
The figure ~\ref{random_graph1} shows a graphs with 4 vertices which means it has Order 4.
Graphs can be either directed or undirected. 
Directed graphs have asymmetrical vertices where as undirected have symmetrical vertices.
An empty graph has no edges.
A complete graph is a graph where all the vertices are connected to the other possible vertices in a graph.\cite{reducible_graph_theory} 
\begin{figure}[H]
	\centering
	\includegraphics[scale = 0.7]{random_graphs/empty_graph.png}
	\caption{Empty graph}
	\label{empty_graph}
\end{figure}
\begin{figure}[H]
	\centering
	\includegraphics[scale = 0.7]{random_graphs/complete_graph.png}
	\caption{Complete graph}
	\label{complete_graph}
\end{figure}
~\ref{empty_graph} shows that empty graph has no edges and ~\ref{complete_graph} shows that a complete graph that all the vertices have edges between the other edges.
$K_n$ represents a complete graph of n vertices.
\newline 
There ways in which a graph of any order $(n)$ can be represented.
One way to represent graphs is by an adjacent matrix(A) which would be an $n \times n$.\cite{reducible_graph_theory}
This method uses a yes or no method to represent the relationship between the vertices.
When there is an edge between two vertices in the matrix we put 1 or else 0.
And the length of the edges would be used.
Considering a symmetric graph.
Each element in the matrix $A_{ij}$ represents the relationship between the vertices.
If there is an edge between two vertices 1 and 2 we would have the elements $A_{12} = A_{21}=1$.
And we only consider graphs without self loops so the diagonal elements would be 0.
Here I define a graph,

\begin{figure}[H]
	\centering
	\includegraphics{random_graphs/random_graph2.png}
	\caption{numbered graph with 4 vertices and 4 edges}
	\label{random_graph2}
\end{figure}
And the matrix representation of this graph would be of a $4\times 4$ considering the order which is 4.
Then we see that the edges are between the number $[(1,2), (2,3), (3,4), (4,1)]$ so the adjacency matrix would be

\begin{equation}
\begin{pmatrix}
0& 1& 0& 1\\
1 & 0 & 1 & 0\\
0 & 1 & 0 & 1\\
1 & 0 & 1 & 0\\
\end{pmatrix}	
\end{equation}
\newline
\newline
There is no unique way to draw a particular graph of Order(n) with m edges.
Although there can be a way of label the graphs.
If we can consider that the adjacency matrix takes in the numbers 1 and 0, this can be a great time to use binary encoding.
Since a binary number can be 1 or 0 depending on the number, we can use this to input in the adjacency matrix. 
In this way for ord(4)\begin{equation}
	\begin{pmatrix}
		0&0_5&1_4&1_3\\
		..&0&0_2&1_1\\
		..&..&0&1_0\\	
		..&..&..&0
	\end{pmatrix}\label{adj_mat}
\end{equation}
and the binary digits in which we are going to use to count/label the graphs so this is given by how i have numbered the upper triangular matrix in the adjacent matrix.

\begin{equation}
	\begin{pmatrix}
		0_5&1_4&1_3&0_2&1_1&1_0
	\end{pmatrix}\label{binary_matrix}
\end{equation}
So converting this to binary we have $2^0+2^1+0+2^3+2^4+0$=1+2+8+16=27.\newline
From the adjacency matrix we can draw the graph depending on the edges, ensuring that the number of binary values is equal to the matrix upper triangular elements.
This is because since for order (3), binary value number one would just be 10.
But we want a number of 3 values instead so we add an extra first digit giving us 010.
Then drawing graph 27 for ord(4) we usually start with defining the binary digit and put it in the adjacency matrix using equation ~\ref{adj_mat} we see that the edges are between [(1,3),(1,4),(2,4),(3,4)] 
\begin{figure}[H]
	\centering
	\includegraphics{random_graphs/graph_27.png}
	\caption{Graph 27 Ord(4)}
	\label{graph 27}
\end{figure}
This gave us a way in which we can define and label our graphs.
And going through all the graphs for order 4, we can notice that they are graphs that look similar.
These graphs of the same order with different structures but same orientations are known as Homomorphic. 
A graph homomorphism f from a graph $G = (V(G),E(G), \phi_G)$ to a graph $H =(V(H),E(H),\phi_H)$
is a mapping from G to H such that each vertex in
$V(G)$ is mapped to a vertex in $V(H)$ with the same label,
and each edge in $E(G)$ is mapped to an edge in $E(H)$. \cite{fan2010graph}
\begin{figure}[H]
	\includegraphics[scale = 0.3]{homomorphic/graph_1.png}
	\includegraphics[scale = 0.3]{homomorphic/graph_2.png}
	\includegraphics[scale = 0.3]{homomorphic/graph_4.png}
	\includegraphics[scale = 0.3]{homomorphic/graph_8.png}
	\includegraphics[scale = 0.3]{homomorphic/graph_16.png}
	\includegraphics[scale = 0.3]{homomorphic/graph_32.png}
	\caption{Homomorphic Graphs Ord(4)}
	\label{Homomorphic}
\end{figure}
This graphs represent one class of homomorphic graphs.
They only have two vertices that have one edge out of the four each.
And each vertex that has an edge has only one edge to another edge.
\newline
\subsubsection*{Ramsey Theory}
Homomorphic graphs help in understanding Ramsey Theory.
Ramsey Theory states given an integer $ r\geq 0$ ,every large enough graph $G = (V(G),E(G),\phi_G)$ contains either  $K_r $ or ${K_b}$ monochromatic clique.\cite{katz2018introduction}
Letting r be red and b represent blue.
In simple terms we can say how many people do I have to invite to a party such that at least r people are strangers or b people are friends.
The graphs have to be complete.
If $G = (V(G), E(G),\phi_G)$, V(G) is a clique if it is a complete graph that is a subset $V(G)$ with order k, then $V'(G)$ is k-clique.\cite{katz2018introduction} 
Ramsey theory deals not really with individual graphs, but with relationships between graphs and other graphs. 
The numbers $R(r,b)$ are known as Ramsey Numbers of r and b.
When we have r = b, for a diagonal case this can be written as R(b).
\begin{figure}[H]
	\includegraphics{random_graphs/clique_2.png}
	\caption{$K_6$ has a monochromatic clique $K_3$ with of blue edges}
	\label{clique}
\end{figure}

\cite{burr1981generalized} talks about generalized Ramsey Theory.
Which tries to understand other structures in Ramsey theory, not just complete graphs.
Here the computational complexity of how to find these numbers is introduced.
Ramsey Numbers are part of NP-hard complete, these are problems that can't be solved simple but they can be verified easily.
Ramsey numbers are very hard to find because the graphs in which we try to find the monochromatic cliques grow super exponential.
If a brute force approach is used one would have to go through $2^{\left(\frac{n^2-n}{2}\right)}$ graphs.
Hence this paper tries to get around this by trying to find other ways of determining them.
An asymptotic formula was introduced that explains how the numbers behave for large n.
And trying to obtain the lower and upper bounds of Ramsey numbers.
A few of these bounds have been found currently.
The paper does not address with finding certain Ramsey numbers only the bounds.
Although these are helpful, it is still very hard to find the numbers even with the bounds.
\section{Computations}
Computation represents process of inputing something and getting an output from that input.
Classical computers use bits to hold information.
This can either be 1 or 0 no in between.
This usually appear in the computer as transistors.
Logic gates are machines that have bits and perform computations.\cite{deutsch1985quantum}
These are the type of computers we use in our everyday lives.
And Quantum computers(QC) store information in quantum bits(qubits).
These can be a superposition of both 1 and 0.
Qubits physically can be thought of as a 2-state system such as a spin-half.
Mathematically this is represented in Dirac notation as $$|1\rangle \& |0\rangle$$
As a vector these are represented as $$|0\rangle \doteq \begin{pmatrix}
	1\\0
\end{pmatrix}$$
$$|1\rangle \doteq \begin{pmatrix}
	0\\1
\end{pmatrix}$$
$|0\rangle$ and $|1\rangle$ represents states.
QC uses principles like superposition and entanglement.
This gives these computers more computational power over the classical ones.
Superposition of a state can be defined as a linear combination of these states.\cite{mcintyre_quantum_2012}
A superposition state of qubits can be expressed as \begin{equation}
|\psi\rangle = \alpha|0\rangle + \beta|1\rangle \doteq \begin{pmatrix}
	\alpha\\\beta
\end{pmatrix} \end{equation}
Where the $\alpha \& \beta$ are complex constants.
And their square of magnitude represents the probability of measuring the associated state, $P_{0}=|\alpha|^2$ and $P_{1}=|\beta|^2$. 
If n particles are added we have $2^n$ states or $2^n$ pieces of information of information. 
For n = 4,
\begin{equation}
	|\psi\rangle =c_1 |00\rangle+c_2 |01\rangle+c_3 |10\rangle+ c_4 |11\rangle
\end{equation}
Representing a new state basis. 
Where in this new state \begin{equation}|00\rangle = |0\rangle\otimes|0\rangle =|0\rangle\langle 0| \doteq
	\begin{pmatrix}
		1\\0
	\end{pmatrix} \begin{pmatrix}
		1&0
\end{pmatrix}\end{equation}
Which returns the new vectors $|00\rangle =	\begin{pmatrix}
	1\\0\\0\\0
\end{pmatrix}$ , $|01\rangle =	\begin{pmatrix}
0\\1\\0\\0
\end{pmatrix}$ ,$|10\rangle =	\begin{pmatrix}
0\\0\\1\\0
\end{pmatrix}$
$|11\rangle =	\begin{pmatrix}
	0\\0\\0\\1
\end{pmatrix}$
Superposition states is the powerful of quantum information processing because the 
amount of information contained in a quantum system grows exponentially with the n  qubits in a system.
And entanglement happens when two or more particles connect to each other.
In QC this introduces \textit{quantum parallelism} where multiple operations are performed at once.
Measuring the state of one particle determines the other's state \cite{mcintyre_quantum_2012}.
A two particle system of entangled states is mathematically expressed as \begin{equation}|\beta_{00} \rangle = a|00\rangle + b|11\rangle\end{equation}
,this is one of the Bell state of a 2-system state.
Bell states are an alternate basis to the couple and uncoupled bases\cite{mcintyre_quantum_2012}.\\
Quantum computers use quantum gates to do their computations\cite{AndrewSteane}.
Quantum gates are used to change the coefficients in qubits without destroying decoherence.
This is because when measured quantum states are sensitive once measured they remain in that state(collapse).
Quantum and logic gates use matrix multiplication to represent their transformations.
For quantum gates Pauli matrices are examples of these.
And these matrices are known as operators which are linear, meaning they act on these states \cite{AndrewSteane,mcintyre_quantum_2012}.
\begin{equation}
\sigma_x =\begin{pmatrix}
	0&1\\1&0
	\end{pmatrix},
	\sigma_y =\begin{pmatrix}
		0&-i\\i&0
	\end{pmatrix},\sigma_z =\begin{pmatrix}
	1&0\\0&-1
	\end{pmatrix}
\end{equation}
Are the Pauli matrices these transform the initial state $\pi$ around their respective basis.
And \begin{equation}H=\frac{1}{\sqrt{2}}
	\begin{pmatrix}
		1&1\\1&-1
	\end{pmatrix}
\end{equation} is the Hadamard gate, this is important because it helps to turn a state into a superposition of the 0 and 1 states.
\\
\\
This article \cite{Deutsch1989} explains quantum computing by expanding on classical computation.
What the challenge at the time was there was lack of quantum computational models.
With this, Deutsch tries how to figure build quantum networks that can do computations like logic gates.
Using the classical as a reference,these quantum gates have to include quantum mechanics principles like superposition.
These quantum network takes qubits as inputs, gates for computation and measurements as outputs.
Deutsch gives a picture of what a quantum machine does and capable of.
And shows how in theory Quantum computers have more computational power over classical ones.
Although these were a great theoretical model, no physical implementations are included.
Even the way in which quantum computers work is not included in the paper. 

\newpage 
\bibliography{sample.bib}
\end{document}

.\newline
\subsection{Quantum Networks}

A logic gate for quantum computation can be represented as a matrix $S$.
For computation to be reversible $S$ has to be unitary or $$S^*S=SS^* =I$$
Depending on the number of particles n, $S$ would have dimensions $2^n\times2^n$.
An inverse of an $S$ matrix is Hermitian.

Here we are introduced to quantum computational networks.
Quantum computations that use quantum coherence are reversible.








