% ------------------------------------------------------------
\documentclass{Assignment}
% ------------------------------------------------------------


\usepackage{pdfpages}
\usepackage{tikz}



% ------------------------------------------------------------
% Formatting
% ------------------------------------------------------------
\usepackage{color}
\usepackage{fullpage}
% ------------------------------------------------------------
\usepackage{float}
\usepackage{graphicx} % Required for inserting images
\usepackage{amssymb}
% ------------------------------------------------------------
% Bibliography
% ------------------------------------------------------------
\usepackage{doi}
\usepackage{hyperref}
\usepackage{usnomencl}
\usepackage[square,sort&compress,comma,numbers]{natbib}
\bibliographystyle{unsrtnat}
\hypersetup{
  colorlinks,
  citecolor=blue,
  linkcolor=red,
  urlcolor=blue}
\usepackage{bibentry}
\usepackage{tikz}
\usepackage{pdfpages}
% ------------------------------------------------------------


% ------------------------------------------------------------

\author{Matsiri Madiba}
\date{April 2025}
\title{Literature Review}
% ------------------------------------------------------------


% ------------------------------------------------------------
\begin{document}

\maketitle
\section{Introduction}
Ramsey theory, named after Frank P. Ramsey, derived from his 1928 paper 'On a problem of formal logic' studying order in large disordered structures \cite{graham1980ramsey}.
He proved a theorem that states that there exists a minimum integer $R(r,b)$ for positive integers $r, b$ such that every graph on $R(r,b)$ vertices contains either a clique of $r$ vertices or an independent set of $b$ vertices \cite{BondyMurty2008}.
A clique is a complete sub-graph that is a subset of a complete graph, and an independent set is a set of vertices with no edges \cite{BondyMurty2008}.	
Ramsey numbers $R(r,b)$ can represent the minimum number of people needed such that either r people are mutual acquaintances or b people are mutual strangers.
Erd\H{o}s and Szekeres rediscovered the theorem in 1935, when they were studying convex polygons formed by points in the plane in combinatorial geometry \cite{BondyMurty2008}.
Ramsey theory has applications in complexity theory, combinatorics and graph theory. 
The goal here is to find Ramsey numbers, particularly $R(5,5)$.
$R(r,b)$ for $r,b$ greater or equal to 5 are currently unknown.
\\\\
This is because computation of Ramsey numbers is computationally intensive, it requires verifying all possible 2-edge-colourings for monochromatic cliques, an NP-hard task with exponential complexity \cite{PhysRevA.93.032301, burr1981generalized}.
The computational challenge of computing Ramsey numbers comes from this NP-hard problem \cite{burr1981generalized}.
These colourings represent all possible ways to label edges with two colours (red and blue).
A monochromatic clique can either be a complete graph of red colour or blue colour.
While analytical methods were used to find $R(4,4)=18$ \cite{GreenwoodGleason1955},
for $R(5,5)$ analytical methods are not feasible.
This number is bounded between 43 \cite{Exoo1993} and 46 \cite{angeltveit2024r55le46}.
It still requires significant computational resources to find the exact value of $R(5,5)$ \cite{spencer1994}.
Classical computers struggle with this complexity.
Quantum computers offer a promising approach to finding Ramsey numbers \cite{PhysRevA.93.032301}.
\\\\
Quantum computers provide a computational advantage over classical ones for this type of problem.
A quantum computer uses superposition and entanglement by exploring multiple 2-edge-colourings simultaneously, offering a computational advantage for combinatorial problems  \cite{Deutsch1989,PhysRevA.93.032301}.
Quantum algorithms such as quantum adiabatic computing and quantum simulated annealing have been proposed for computing Ramsey numbers \cite{gaitan2012ramsey, PhysRevA.93.032301}.
\\
This paper reviews classical and quantum computational methods for computing Ramsey numbers.
\section{Classical Computation of Ramsey Numbers}
\subsection{Graph Theory}
Graph theory is a study of mathematical structures called graphs, which aims to understand relationships between objects. 
A graph is a structure $G =\left(V,E\right)$ made up of a vertex set $V$ and an edge set $E$.
$V = \{v_1,v_2,v_3,...,v_n$\} has elements called vertices, and the edge set $E$ has pairs of objects taken from $V$ called edges \cite{BondyMurty2008}.
The order of a graph is the number of vertices.
Edges are unordered pairs $e_{ij} = \left\{v_i,v_j\right\}$ for undirected graphs or ordered pair $e_{ij} = (v_i,v_j)$ for directed graphs with $i\neq j$  \cite{BondyMurty2008}.
Directed graphs have asymmetrical edges $e_{ij} \neq e_{ji}$ whereas undirected have symmetrical edges $e_{ij} = e_{ji}$.
An empty graph has no edges, whose complement is a complete graph with possible edges between all pairs of vertices.
Graph theory has applications in computer science, mathematics and social network modeling.
\\\\
Graphs can be visualized by representing vertices as points and edges as lines that connect the vertices.
Let $G = (V,E)$ be a graph with $V = \{v_1, v_2, v_3, v_4\}$
\begin{figure}[H]
	\centering
	\includegraphics[scale = 0.7]{random_graphs/figure_1.png}
	\caption{Random graph with four vertices and two edges}
	\label{random_graph}
\end{figure}
For simplicity the vertices can be labeled.
For figure \ref{random_graph} the graph is given by the sets  $V $ and $E=\{e_{12},e_{14}\}.$
\begin{figure}[H]
	\centering
	\includegraphics[scale = 0.7]{random_graphs/empty_graph.png}
	\caption{Empty graph}
	\label{empty_graph}
\end{figure}
The graph in figure \ref{empty_graph} is given by the sets $V$ and $E=\{\}.$
\begin{figure}[H]
	\centering
	\includegraphics[scale = 0.7]{random_graphs/complete_graph.png}
	\caption{Complete graph with four vertices }
	\label{complete_graph}
\end{figure}
The graph in figure \ref{complete_graph} is given by the sets  $V $ and $E=\{e_{12},e_{13},e_{14},e_{23},e_{24},e_{34}\}.$ 
Figure \ref{empty_graph} shows a graph with no edges and figure \ref{complete_graph} is graph that has all the vertices with possible edges connecting to other vertices.
A complete graph with n vertices has ${\frac{n^2-n}{2}}$ edges.
For a two colouring graph, there would be 2 vertices from n. 
$$n C 2 =\frac{n!}{(n-2)!n!}= {\frac{n^2-n}{2}}$$
This is because each vertex pair forms a unique undirected edge.
Such representations are crucial in Ramsey theory for visualizing cliques and independent sets \cite{BondyMurty2008}.
\\\\
Another way to represent a graph is with an adjacency matrix.
This is a table that contains information on the relationship represented by the set $ E$.
Each column and row represents a vertex in the graph.
Each element in the matrix $A_{ij}$ represents the relationship between the vertices.
If there is an edge in the edge set $E$ then the entry in the matrix between the vertices would be 1, else equal 0.
\begin{equation}
	A=	\begin{tabular}{c|c|c|c|c|c}\hline
		&$v_1$&$v_2$&$v_3$&$v_j$&$v_n$\\\hline
		$v_1$&$e_{11}$&$e_{12}$&$e_{13}$&$e_{1j}$&..\\
		$v_2$&$e_{21}$&$e_{22}$&$e_{23}$&$e_{2j}$&..\\
		$v_3$&$e_{31}$&$e_{32}$&$e_{33}$&$e_{3j}$&..\\
		$v_4$&$e_{41}$&$e_{42}$&$e_{43}$&$e_{4j}$&..\\
		$v_4$&$\vdots$&$\ddots$&$\ddots$&$\vdots$&..\\
		$v_n$&$e_{n1}$&$e_{n2}$&$e_{n3}$&..&$e_{nn}$\\
	\end{tabular}
\end{equation}
Filling the elements by either 1 or 0 depending on if there is an element in the edge set that matches with the elements of the matrix.
If there is an edge between two vertices $v_1 ,v_2$ we would have the elements $A_{12} = A_{21}=1$. 
Graphs have no unique way of representing them visually.
In the set of graphs of n vertices some have identical structure.
Their edge set would be the same if their vertices were relabeled.
These graphs of the same order with different structures but same orientations are known as Homomorphic. 
A graph homomorphism from a graph $G = (V_G,E_G)$ to a graph $H =(V_H,E_H)$
is a mapping from G to H such that each vertex in
$V_G$ is mapped to a vertex in $V_H$ with the same label,
and each edge in $E_G$ is mapped to an edge in $E_H$ \cite{fan2010graph}.
\begin{figure}[H]
	\centering
	\begin{tabular}{|c|c|c|}\hline
		\hline
		\includegraphics[scale=0.3]{homomorphic/graph_1.png} &
		\includegraphics[scale=0.3]{homomorphic/graph_2.png} &
		\includegraphics[scale=0.3]{homomorphic/graph_4.png} \\\hline
		\hline
		\includegraphics[scale=0.3]{homomorphic/graph_8.png} &
		\includegraphics[scale=0.3]{homomorphic/graph_16.png} &
		\includegraphics[scale=0.3]{homomorphic/graph_32.png} \\
		\hline
	\end{tabular}
	\caption{Homomorphic Graphs of four vertices, only one edge between them}
	\label{Homomorphic}
\end{figure}
These graphs represent graphs with same structure and properties.
Only two vertices have an edge between them.
This is going to help in understanding Ramsey theory because they reduce the number of graphs needed to be verified.
\subsection{Ramsey Theory}
Ramsey Theory states given integers $ r,b\geq 1$, every large enough graph $G = (V,E)$ contains a monochromatic clique of either $r$ or $b$ vertices.  \cite{katz2018introduction}.
Monochromatic clique is a complete sub-graph with the same colour that is a subset of a larger complete graph \cite{BondyMurty2008}.
\begin{figure}[H]
	\centering
	\includegraphics[scale = 0.5]{random_graphs/clique_2.png}
	\caption{The graph has clique $K_3$}
	\label{clique}
\end{figure}
The graph in figure \ref{clique} has a monochromatic clique with blue edges.
In Ramsey theory complete graphs are considered \cite{burr1981generalized}.
$K_n$ is a notation to represent a complete graph of n vertices.
Ramsey number is the minimum number of vertices $R(r,b)$ such that every 2-colouring of the edges of the complete graph $K_n$ contains a clique of order r or a clique of order b.
\\\\
Ramsey numbers are very hard to find because the graphs in which we try to find the monochromatic cliques increase exponentially.
For an order n graph, the total amount of graphs would be $2^{\left(\frac{n^2-n}{2}\right)}$.
They are increasing with order $\sim 2^{\frac{n^2}{2}}$.
\begin{figure}[H]
	\centering
	\includegraphics[scale = 0.5]{plot.png}
	\caption{Plot of amount of graphs per n-vertices}
	\label{plot}
\end{figure}
For example for order 4, we would have a total of 64 graphs and for order 5 we have 1024 graphs.
Going from this noticing that $R(3,3) = 6$ the total number of graphs of order 6 is $32768$.
This difference is not visible on the plot \ref{plot} because of how these graphs exponentially increase.
The orders of magnitude for $n= 10$ are in the $10^{12}$ which is a huge number.
\begin{figure}[H]
	\centering
	\includegraphics[scale = 0.5]{plot_0.png}
	\caption{Logarithmic scale of the increase of n vertex graphs}
	\label{plot_0}
\end{figure}
In the logarithmic scale \ref{plot_0} the increase is shown by a quadratic growth.
The total amount of graphs increase fast.
To find these clique we have to go through each graph.
These graphs grow larger with increase in n.
A few Ramsey numbers are known.
\\\\
The highest known Ramsey number is $R(4,4)$ which was proved to be $18$ \cite{GreenwoodGleason1955}.
This proof was done by using the bounds.
Ramsey numbers are bounded between certain numbers.
It was shown that $R(4,4) > 17$ by checking if the graphs of 17 vertices cannot have a monochromatic clique of 4 vertices. 
Using the proof,
\begin{equation}
	R(r ,b) \geq R(r-1,b)+ R(r,b-1) 
	\label{theorem_2}
\end{equation}
$$R(4,4) \geq R(3,4)+R(4,3) = 2R(3,4)$$
$R(3,4) $ was found to be 9, from equation \eqref{theorem_2} it was proved that $R(4,4) \leq 18$ \cite{GreenwoodGleason1955}.
Hence combining the results it was proved that $R(4,4) = 18$.
The same result cannot be shown for 5th Ramsey number.
$R(5,5)$ was found to be between 43 \cite{Exoo1993} and 46 \cite{angeltveit2024r55le46}.
For $n= 43$, the total number of graphs is $2^{903} \approx 10^{271}$ and for $n=46$ we have $2^{1035}\approx10^{312}$.
This makes checking the graphs or analytical proofs not feasible with traditional methods.
\\\\
Methods of finding Ramsey numbers were introduced, like generalized Ramsey theory \cite{burr1981generalized}. 
Generalized Ramsey theory does not only consider complete graphs unlike classical.
At first Ramsey numbers were found using analytical methods.
Proofs and combinatorial methods of checking if graphs do not have a certain clique of r amount of graphs is not feasible.
Analytic approaches start to fail since the orders of magnitude are impossible to work with.
Since as n grows, it becomes exponentially harder as shown by figure ~\ref{plot}.
\\
Computers have been used to find bounds of Ramsey numbers.
Methods like Linear programming can reduce the number of graphs considered \cite{angeltveit2024r55le46}.
This was done by reducing the number of cases to check by identifying feasible solutions. 
With known Ramsey numbers and equation ~\ref{theorem_2} bounds can be determined.
We now know $R(5,5)$ is between 43 and 46.
It is still extremely hard to determine the exact value.
With the fastest classical computers available in the world, it would take a lot of years to go through every graph.
Quantum computers can be considered to compute Ramsey numbers.
\\\\
The third section will deal with quantum computations of Ramsey numbers.
It will deal with why quantum computers have more computational power over classical ones.
Which would then lead to using them to possibly find Ramsey numbers.
\section{Quantum Computation of Ramsey Numbers}
\subsection{Quantum Computers}



Classical computers use bits to hold information.
Bits represent a single binary number.
They can either be 1 or 0.
Logic gates use bits and to perform computations \cite{deutsch1985quantum}.
Logic gates are electronic circuits that take in bits as input and produce a bit as output based on a logical operation \cite{Jaeger1997}.\\
Quantum computers store information in quantum bits (qubits).
A spin-half system is a system that has 2 states spin up and spin down similar to qubits.
Mathematically qubits are represented  as in Dirac notation \begin{center}
	0 bit $=|0\rangle $ \\1 bit $=  |1\rangle$
\end{center}
As a vector these are represented as 
$$|0\rangle = \begin{pmatrix}
	1\\0
\end{pmatrix}$$
$$|1\rangle = \begin{pmatrix}
	0\\1
\end{pmatrix}$$
Quantum Computing uses principles like superposition and entanglement.
This gives these computers more computational power over the classical ones.
Superposition of a state can be defined as a linear combination of these states \cite{mcintyre_quantum_2012}.
A superposition state of qubits can be expressed as \begin{equation}
	|\psi\rangle = \alpha|0\rangle + \beta|1\rangle = \begin{pmatrix}
		\alpha\\\beta
		\label{superposition}
\end{pmatrix} \end{equation}
The constants $\alpha $ and $ \beta$ are complex constants.
Their square of magnitude represents the probability of measuring the associated state.
To find the probability of finding a state $|\psi\rangle$ of equation \eqref{superposition} in $|0\rangle$ state.
I would say
$$P_0=|\langle0|\psi\rangle|^2 = \left| \begin{pmatrix}
	1&0
\end{pmatrix}^*\begin{pmatrix}
	\alpha\\\beta
\end{pmatrix} \right|^2 = |\alpha|^2$$
Where $\langle 0|$ is the complex conjugate of $|0\rangle$.
Superposition states make quantum computers powerful, because the amount of information contained in a quantum system grows exponentially with the n qubits in a system.
If n particles are added we have $2^n$ states or $2^n$ pieces of information. 
For n = 2 we have 4 different states which are,
\begin{equation}
	|\psi\rangle =c_1 |00\rangle+c_2 |01\rangle+c_3 |10\rangle+ c_4 |11\rangle
\end{equation}
Where in this new state \begin{equation}|00\rangle = |0\rangle\otimes|0\rangle  =	\begin{pmatrix}
		1\\0\\0\\0
\end{pmatrix}\end{equation}
The other vectors are given by,
$$01\rangle =	\begin{pmatrix}
	0\\1\\0\\0
\end{pmatrix}|10\rangle =	\begin{pmatrix}
	0\\0\\1\\0
\end{pmatrix}|11\rangle =	\begin{pmatrix}
	0\\0\\0\\1
\end{pmatrix} $$
This represents a 2-qubit superposition state with the coefficients $c_i$ are complex and like the 1-qubit system they tell us the probability of measuring their respective states.
\\\\
Quantum entanglement is when quantum states of multiple particles are connected, such that the state of one particle cannot be described without considering the states of the other \cite{Horodecki_2009}.
This introduces \textit{quantum parallelism} in quantum computing, where multiple operations are performed at once \cite{mcintyre_quantum_2012}.
A two particle system of entangled states is mathematically expressed as
\begin{equation}
	|\psi \rangle = a|00\rangle + b|11\rangle
	\label{entagled_state}
\end{equation}
The equation \eqref{entagled_state} is one of the Bell states of a 2-system state.
Bell states are an alternate basis to the couple and uncoupled bases \cite{mcintyre_quantum_2012}.
Bell states are quantum states of two qubits that represent examples of quantum entanglement in quantum computing.
When one of the qubits in the states is measured and takes on a certain value, the other one is going to take the same value.
This is because in quantum mechanics a state can be measured once, measurement collapses a state to classical \cite{mcintyre_quantum_2012}.
Quantum computers use quantum gates to do their computations \cite{AndrewSteane}.

Quantum gates are used to change the coefficients in qubits without destroying decoherence.
Destroying decoherence means the qubits lose their quantum properties.
This can be because they interact with the environment.\\
Quantum states are sensitive, measuring them causes them to be in a classical state.
Quantum and logic gates use matrix multiplication to represent their transformations.
Pauli matrices are examples of quantum gates.
These matrices are known as operators which are linear, meaning they act on these states \cite{AndrewSteane,mcintyre_quantum_2012}.
\begin{equation}
	\sigma_x =\begin{pmatrix}
		0&1\\1&0
	\end{pmatrix},
	\sigma_y =\begin{pmatrix}
		0&-i\\i&0
	\end{pmatrix},\sigma_z =\begin{pmatrix}
		1&0\\0&-1
	\end{pmatrix}
\end{equation}
These are Pauli matrices and they rotate the initial state $\pi$ about their respective basis.
If a state was in a certain initial state with a phase 0, then $\sigma_x$ acts on the state, it would later be in $\pi$ radians about the x-axis.
\begin{equation}U_H=\frac{1}{\sqrt{2}}
	\begin{pmatrix}
		1&1\\1&-1
	\end{pmatrix}
\end{equation} is the Hadamard gate, this is an important gate that helps turn a state into a superposition of the 0 and 1 states.
If an initial state is 0, 
\begin{equation}
	U_H|0\rangle = \frac{1}{\sqrt{2}}
	\begin{pmatrix}
		1&1\\1&-1
	\end{pmatrix}\begin{pmatrix}
		1\\0
	\end{pmatrix}=\begin{pmatrix}
		\frac{1}{\sqrt{2}}\\
		\frac{1}{\sqrt{2}}
	\end{pmatrix}=\frac{1}{\sqrt{2}}(|0 \rangle + |1\rangle)
\end{equation} 
If we start with the initial state 1,
$$U_H|1\rangle = \frac{1}{\sqrt{2}}(|0 \rangle - |1\rangle)$$
giving a superposition of the states.
There is a Controlled-NOT(CNOT) gate.
\\
The CNOT gate introduces entanglement.
The CNOT gate is represented as 
\begin{equation}
	U_X = \begin{pmatrix}
		1&0&0&0\\
		0&1&0&0\\
		0&0&0&1\\
		0&0&1&0
	\end{pmatrix}
\end{equation}
Combining CNOT gate with Hadamard gate produces entanglement.
A CNOT gate has two input qubits, referred to as the control and target qubits, and two output qubits. \cite{mcintyre_quantum_2012}

\begin{figure}[H]
	\centering
	\includegraphics{screenshot001}
	\caption{Quantum gate with Hadamard and CNOT gate}
	\label{entanglement}
\end{figure}
Figure \ref{entanglement} the input state is $|0\rangle|0\rangle$.
The Hadamard gate will transform the upper $|0\rangle$ meaning,
$$U_H|0\rangle = \frac{1}{\sqrt{2}}(|0 \rangle + |1\rangle)$$
Then with the bottom target $|0\rangle$ we have
$$\frac{1}{\sqrt{2}}(|0 \rangle_C + |1\rangle_C)|0\rangle_T$$
$$=\frac{1}{\sqrt{2}}(|0 \rangle_C|0 \rangle_T + |1\rangle_C|0 \rangle_T)$$
Then CNOT ($U_X$) acting on this state we have 
$$U_X\frac{1}{\sqrt{2}}(|0 \rangle_C|0 \rangle_T + |1\rangle_C|0 \rangle_T) $$
$$ =\frac{1}{\sqrt{2}}\left(|00\rangle + |11\rangle\right)$$
Equation \eqref{entagled_state} with $a=b = \frac{1}{\sqrt{2}}$.
Showing how we can start from two qubits and entangle them using the Hadamard and CNOT gates.
This shows how in theory quantum computers have more computational power over classical ones \cite{Deutsch1989}.
Another powerful quantum mechanics concept is the idea of Hamiltonian.
\\\\
A Hamiltonian is an operator that is equal to the total energy of a system \cite{mcintyre_quantum_2012}.
The Hamiltonian governs time evolution of a system.
This is described the Schr$\ddot{o}$dinger equation \cite{mcintyre_quantum_2012}.
\begin{equation}
	i \hbar\frac{\partial}{\partial t}|\psi(t)\rangle = H(t)|\psi(t)\rangle
	\label{Schrodinger_equation}
\end{equation}
where $\hbar$ is the reduced Planck's constant, $H$ is the Hamiltonian and $|\psi(t)\rangle$ is the quantum state.
The solution of equation \eqref{Schrodinger_equation} is given by 
\begin{equation}
	|\psi(t)\rangle = e^{-i\frac{ H\cdot t}{\hbar}}|\psi(0)\rangle
\end{equation}
$U(t)=e^{-i\frac{ H\cdot t}{\hbar}} $ is the time-evolution operator a unitary operator that changes the state over time \cite{UQS}.
In quantum computing the Hamiltonian is used to describe the dynamics of qubits and to design quantum gates, which are unitary transformations that manipulate quantum states. \cite{Deutsch1989} 
Quantum gates are implemented by applying specific Hamiltonian to a quantum system for a specific time using $U$.
The Hamiltonian depends on the type of problem defined.
\subsubsection*{Quantum Computer and Ramsey Numbers}
The reason a quantum computer is preferred over classical is because some algorithms can outperform in certain tasks. 
Simon's algorithm showed that for a type of searching problem, quantum computers outperform classical computers \cite{doi:10.1137/S0097539796298637}.
Quantum superposition checks all inputs at once and interference is used to see the correct output.
Quantum interference is when the probability magnitudes of the states are decreased using destructive interference and increased using constructive interference.
\\\\
There is another algorithm called Quantum Annealing which was extended from a classical Simulated Annealing.
This method takes advantage of how systems always go to the lowest energy.
This algorithm can be used to solve combinatorial optimization problems \cite{AndrewSteane}.
Using superposition and entanglement, the quantum annealer is able to calculate all potential solutions at once, which speeds up the 
calculation process \cite{AndrewSteane}.
\\\\
Finding Ramsey numbers can be mapped as combinatorial optimization problem \cite{PhysRevA.93.032301}.
The algorithm uses adiabatic quantum computing, evolving the system from a simple initial Hamiltonian to the problem's Hamiltonian.
If the Hamiltonian is defined such that the ground state corresponds to a coloring without a red $k_r$ or blue $k_b$.
The Hamiltonian of the problem is given by
\begin{equation}
	H = -\frac{1}{2}\omega \sigma_z + I_2 \otimes H_Q+c\sigma_x\otimes A
	\label{hamiltonian}
\end{equation}
it contains Pauli's x and z matrices, with $I_2$ being identity matrix of size $2 x 2$ , c is a coupling constant, $\omega$ is the frequency of the qubit.
$$H_Q =|0\rangle\langle0|\otimes[\epsilon_0(|0\rangle\langle0|^{\otimes L})] + |1\rangle\langle 1|\otimes H_P$$
$\epsilon_0$ is a parameter set as a reference point to the ground state energy $H_P$.
The operator $$A = \sigma_x \otimes H_d^{\otimes L}$$
$H_d$ is the Hadamard matrices , $L = \frac{n(n-1)}{2}$.
The idea was to see if ground state energy of $H_P$ is equal to zero or not.
A certain qubit is prepared as an initial state.
Then apply the Hamiltonian in equation \eqref{hamiltonian} time evolution. 
Using the known Ramsey numbers bounds.
If the ground state energy is zero for a certain n, then the number is not Ramsey number. 
This method is still theoretical, but may help in finding $R(5,5)$.

\section{Conclusion}
Ramsey numbers define the smallest number such that every 2-edge-colouring of $K_n$ contains a monochromatic clique of either blue $K_b$ or red $K_b$.
Graph theory helps in understanding the complexity of finding Ramsey numbers.
The total number of graphs to be searched for cliques increases exponentially.
Binary encoding enumerates the total number of graphs.
$R(5,5)$ is bounded between the numbers $43$ and $46$.
Despite these bounds, the exact number is still hard to compute analytically.
This is also difficult using classical computers.
Thus, quantum computing provides a promising approach for finding $R(5,5)$ using adiabatic quantum computing \cite{PhysRevA.93.032301}.
This showed that this quantum algorithm solves this faster than the classical.
\\\\
\cite{rietsche2022quantum} gives an overview of quantum computing.
Seeing that many fields might need quantum computers to help solve problems, a knowledge of it is needed.
These organizations don’t know how to leverage quantum computing practically.
Here this article addresses this by providing three ways in which a quantum computer operates.
Quantum annealing is used an analog of how they operate.
Also by providing different applications that can be used like optimization.
Search and graph is used as one of examples of optimization applications.
After defining these, another way to deal with this was proposed is to introduce research areas in different places.
Different way of looking at this paper is that it assumes breakthroughs in error correction.
\newpage
\section*{Comments}




From reading the paper \cite{burr1981generalized}, here I see they speak on the computational complexity of Ramsey Numbers. 
I have a little trouble in understanding Ramsey numbers as a class of $\sum_{2}^{p}$. 
I also do not get the notation or how Ramsey numbers are part of this class.
The mathematical notation in this paper is the most confusing, like what $W_{1,n}$ n-spoked wheel represent.
\\
\\
\cite{Deutsch1989}
A complete set C of compatible observables for $\mathcal{P}$ is chosen, including an input 
observable $\hat{I}$ and an observable $\hat{h}$ that is independent of $\hat{I}$ and has spectrum \{0,1\}, the halt flag. 
This part in the paper is confusing.
A halt flag is introduced not sure of its importance but im confused by this.
I do not understand how from this set what compatible observables are chosen.
\\
\\
I notice I am very bad at getting the main ideas from papers.
And even the main aims of the papers.
\\
\\
In this paper \cite{doi:10.1137/S0097539796298637}, asks 'if a function is invariant under XOR-mask' I do not understand this part.
Not fully sure what an "XOR-mask"
Majority of this paper speaks about "Oracle".
I feel like the reason i dont understand these parts is because the paper is more computer science based/ the notation of computer science is used and my background is poor. 
\\\\
In the paper \cite{PhysRevA.93.032301} im confused on how they define the Hamiltonian. 
On why it was defined in this manner.
They mention a probe qubit, i dont know what it represents.
\subsubsection*{AI comments}
Critique of Literature Review
This critique addresses the English and academic writing in your LaTeX document, focusing on areas for improvement to enhance clarity, precision, and academic tone.
\\
\begin{enumerate}

\item Overall Impression
\\
The document outlines a relevant topic, moving from classical graph theory and Ramsey numbers to the potential application of quantum computing. However, the writing quality is significantly impacted by numerous grammatical errors, awkward phrasing, imprecise technical explanations, and an inconsistent academic tone. These issues make the text difficult to read and detract from its credibility.
\\
\item English Grammar and Sentence Structure
\\
This is the most prominent area needing improvement.
\\
Frequent Errors: The text contains numerous grammatical mistakes, including issues with subject-verb agreement, missing articles (e.g., "A graph is set"), incorrect prepositions, incorrect use of pronouns (e.g., "If their labeled vertices"), and run-on sentences or comma splices.
\\
Awkward Phrasing: Many sentences are grammatically incorrect or structured in a way that is unnatural or overly simplistic for academic writing (e.g., "To find Ramsey numbers we need check," "Reason is that...").
\\
Informal Language: The use of informal expressions (e.g., "a lot of time," "very huge jump," "a couple," "go through") undermines the academic tone.
\\
\item Academic Tone and Vocabulary
\\
Inconsistent Tone: The writing shifts between attempting a formal style and using informal language. Maintaining a consistent, objective, and formal tone is crucial for academic work.
\\
Imprecise Vocabulary: Concepts are often described using general or vague terms (e.g., "hard," "more computational power") rather than specific academic vocabulary (e.g., "intractable," "computationally expensive," "exponential speedup").
\\
\item Clarity and Precision of Technical Concepts
\\
Accuracy and clarity in explaining technical concepts are essential.
\\
Ramsey Number Definition: The initial definition is slightly inconsistent with the later, more precise definition focusing on 2-colourings of complete graphs. Ensure a single, clear, and accurate definition is used throughout.
\\
Graph Theory Fundamentals: There are fundamental errors in basic graph theory definitions:
\\
The description of directed vs. undirected edges is incorrect.
\\
The graphs labeled as "Homomorphic Graphs" are actually isomorphic. Homomorphism is a different concept.
\\
Binary Encoding Example: The explanation of binary encoding is confusing. The connection between the matrix notation, the subscripts, and the resulting binary vector and edge set is unclear and appears inconsistent.
\\
Complexity Classification: The statement classifying finding Ramsey numbers as NP-complete based on the difficulty of finding cliques is a simplification that is potentially inaccurate in a strict computational complexity sense.
\\
Ramsey Bounds: The presentation of the recursive inequality for Ramsey numbers seems to use the symbol incorrectly or implies a derivation that is not standard.
\\
Quantum Computing Concepts: Several explanations of quantum mechanics and quantum computing concepts contain errors or are misleading:
\\
The probability calculation shown is incorrect.
\\
The justification for entanglement correlation is inaccurate.
\\
The description related to decoherence and quantum gates is imprecise.
\\
The explanation of Simon's algorithm and Quantum Annealing is oversimplified.
\\
The discussion of the specific Hamiltonian is too brief and uses specialized terms without sufficient explanation for a general reader.
\\
\item Structure and Flow
\\
Logical Sections: The division into sections (Introduction, Classical Computation, Quantum Computation, Conclusion) provides a reasonable overall structure.
\\
Paragraph Transitions: The flow between paragraphs within sections can be abrupt due to weak or missing transition sentences.
\\
Repetition: Some points, such as the difficulty of finding R(5,5), are repeated unnecessarily.
\\
6. Use of Citations
\\
Citations are included and linked using  cite{...}, which is appropriate.
\\
However, ensure that the statements linked to citations accurately reflect the content and claims made in the cited sources, especially concerning technical details and complexity.
\\
Recommendations for Improvement
\\
To improve the document, focus on the following steps:
\\
Thorough Proofreading: Correct all grammatical errors, typos, and awkward sentence structures.
\\
Enhance Academic Tone: Use formal language and avoid informal expressions.
\\
Refine Technical Explanations: Ensure all definitions (graph theory, quantum mechanics) are precise and accurate. Correct the errors identified in graph theory definitions, probability calculations, and quantum concepts.
\\
Clarify Complex Ideas: Rework explanations that are confusing or overly simplified (binary encoding, quantum algorithms, the specific Hamiltonian). Provide necessary context.
\\
Improve Flow: Use transition sentences to create smoother connections between ideas and paragraphs.
\\
Verify Accuracy: Double-check all technical claims, especially regarding complexity, bounds, and the specifics of the quantum algorithm, against reliable academic sources.
\\
Ensure Citation Accuracy: Confirm that the text accurately represents the information from the cited sources.
\\
Addressing these points will significantly enhance the clarity, accuracy, and academic quality of your literature review.

\end{enumerate}
\section*{Previous draft comments}
In my previous Literature review i failed to make connections between my paragraphs.
Majority of them had no structure.
And no relations between the sentences i had with the paragraphs.
While also i notice i made many spelling and grammar mistake.
\newpage
\bibliography{sample.bib}
\end{document}







