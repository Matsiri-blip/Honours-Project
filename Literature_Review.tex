% ------------------------------------------------------------
\documentclass[15pt, a4paper]{Assignment}
% ------------------------------------------------------------


% ------------------------------------------------------------
% Formatting
% ------------------------------------------------------------
\usepackage{color}
\usepackage{fullpage}
% ------------------------------------------------------------
\usepackage{float}
\usepackage{graphicx} % Required for inserting images
\usepackage{amssymb}
% ------------------------------------------------------------
% Bibliography
% ------------------------------------------------------------
\usepackage{doi}
\usepackage{hyperref}
\usepackage{usnomencl}
\usepackage[square,sort&compress,comma,numbers]{natbib}
\bibliographystyle{unsrtnat}
\hypersetup{
  colorlinks,
  citecolor=blue,
  linkcolor=red,
  urlcolor=blue}
\usepackage{bibentry}
\usepackage{tikz}
\usepackage{pdfpages}
% ------------------------------------------------------------


% ------------------------------------------------------------

\author{Matsiri Madiba}
\date{April 2025}
\title{Literature Review}
% ------------------------------------------------------------


% ------------------------------------------------------------
\begin{document}
\maketitle

\section*{Introduction}
Our main goal is to find Ramsey numbers.
Ramsey numbers $R(5,5)$, and succeeding are currently not known.
These are the smallest numbers needed to have order in a large system.
They explain minimum number of people needed such that at least r people are strangers or b people are friends.
 Frank Ramsey came up with Ramsey Theory in 1928.
Ramsey's Theorem was rediscovered by Erd$\ddot{o}$s and Szekeres in 1935 while investigating a problem on combinatorial geometry \cite{BondyMurty2008}.									
It has applications in complexity theory, combinatorics and graph theory. 
Here we will explore this in graph theory and complexity theory in the first section.
Graphs are used to understand Ramsey theory.
It deals not really with individual graphs, but with relationships between graphs.							
\\\\
Finding Ramsey numbers is extremely difficult.
It would require a lot of computing power to find $R(5,5)$ \cite{spencer1994}.
To find Ramsey numbers we need check for cliques in all possible colourings.
This makes Ramsey Numbers part of a class called NP-hard complete  \cite{burr1981generalized}. 
These are problems that take a lot of time to solve and would take as much time to verify the solutions. 
Because it would take a long time for a classical one,
Quantum Computers can be used to find these Ramsey numbers \cite{PhysRevA.93.032301}.
Quantum computers provide a computational advantage that classical ones cannot.
The main idea is to find Ramsey numbers using Quantum computers.
Different quantum algorithms can be used to calculate Ramsey Numbers as in \cite{gaitan2012ramsey, PhysRevA.93.032301}.
\newline
\\
Deutsch \cite{Deutsch1989} builds on universal quantum computer by \cite{deutsch1985quantum} from classical computing to quantum mechanics. 
The development of quantum computation theory is important in building quantum computers \cite{Deutsch1989}.
Quantum computer can perform powerful computations using superposition and entanglement.
The concept of quantum computing developed in the early 1980s by Richard Feynman  \cite{feynman1982simulating}. 
There is a quantum computing algorithm that can show how fast these computers are	\cite{doi:10.1137/S0097539796298637}.
Using the article \cite{shor1994algorithms} he explores the development of quantum polynomial-time algorithms.
\\\\
In the first section will be on Ramsey numbers using classical computations. 
This would help understanding Ramsey theory visually using graphs.
While in the second section will be using quantum computation to try and find Ramsey numbers.
\section{Classical Computation of Ramsey Numbers}
\subsection*{Graph Theory}
Graph theory is a study of graph which aims to understand relationships between points. 
A graph is set of ordered pair of objects $G = \{V(G),E(G)\}$ called vertices $V(G)$, a pair of objects taken from $E(G)$ called edges \cite{BondyMurty2008}.
And $V(G)$ = \{$v_i$\} and $E(G)=\{e_{ij}\}$ $i,j$ are indices that can run up to the number of vertices in the graph. 
An order of a graph is given by the number of vertices in the graph.
If an there is an edge between vertices $v_1 ,v_2$, this would be given by an element in $E(G)$ as $e_{12}=v_1v_2$ \cite{BondyMurty2008}.
Graphs can be either directed or undirected. 
Directed graphs have asymmetrical vertices where as undirected have symmetrical vertices.
For a directed graph $e_{12} = e_{21}$, but for an undirected graph $e_{12} \neq e_{21}$.
An empty graph has no edges, the edge set $E(G)$ would be empty.
And its complement would be a complete graph.
This is a graph where all the vertices are connected to the other possible vertices in a graph \cite{reducible_graph_theory}.
\\\\
One way to represent a graph is by representing points as vertices and lines joining the points as edges.
Let $G = (V(G),E(G))$ be a graph with $V(G) = \{v_1, v_2, v_3, v_4\}$ and $ E(G)=\{e_{ij}\}.$
\begin{figure}[H]
	\centering
	\includegraphics[scale = 0.7]{random_graphs/figure_1.png}
	\caption{Random graph}
	\label{random_graph}
\end{figure}
For ~\ref{random_graph} the graph would be $V = \{v_1, v_2, v_3, v_4\}$ and $E=\{v_1v_2,v_1v_3,v_1v_4\}.$
\begin{figure}[H]
	\centering
	\includegraphics[scale = 0.7]{random_graphs/empty_graph.png}
	\caption{Empty graph}
	\label{empty_graph}
\end{figure}
For ~\ref{empty_graph} the graph would be $V = \{v_1, v_2, v_3, v_4\}$ and $E=\{\}.$
\begin{figure}[H]
	\centering
	\includegraphics[scale = 0.7]{random_graphs/complete_graph.png}
	\caption{Complete graph}
	\label{complete_graph}
\end{figure}
And lastly for \ref{complete_graph} the graph would be $V = \{v_1, v_2, v_3, v_4\}$ and $E=\{v_1v_2,v_1v_3,v_1v_4,v_2v_3,v_2v_4,v_3,v_4\}.$\\\\
~\ref{empty_graph} shows a graph with no edges and ~\ref{complete_graph} is graph that has all the vertices with possible edges connecting to other vertices.
\\\\
Another way to represent a graph is with an adjacency matrix.
This is a table that contains information on the relationship represented by the set $ E(G)$.
Each column and row represents a vertex in the graph.
Each element in the matrix $A_{ij}$ represents the relationship between the vertices.
If there is an edge in the edge set $E(G)$ then the entry in the matrix between the vertices would be 1 else equal 0.
Considering a symmetric graph.
If there is an edge between two vertices $v_1 ,v_2$ we would have the elements $A_{12} = A_{21}=1$.
\begin{figure}[H]
	\centering
	\includegraphics[scale = 0.5]{random_graphs/random_graph2.png}
	\caption{Graph with 4 vertices}
	\label{random_graph2}
\end{figure}
And the matrix representation of this graph would be of a $4\times 4$ because its of order 4.
Then we see that the edges are between the number $[(1,2), (2,3), (3,4), (4,1)]$ so the adjacency matrix would be

\begin{equation}
\begin{pmatrix}
0& 1& 0& 1\\
1 & 0 & 1 & 0\\
0 & 1 & 0 & 1\\
1 & 0 & 1 & 0\\
\end{pmatrix}	
\end{equation}
\newline
\newline
There is no unique way to draw graphs.
Although there can be a way of label the vertices of a graph.
Adjacency matrix takes in 1 or 0, so binary can be used.
Numbering each graph using binary can make things easier.
If for a directed graph the adjacency matrix would be,\begin{equation}
	\begin{pmatrix}
		0&0_5&1_4&1_3\\
		0_5&0&0_2&1_1\\
		1_4&0_2&0&1_0\\	
		1_3&1_1&1_0&0
	\end{pmatrix}\label{adj_mat}
\end{equation}
then we can extract a binary number with the entries of the matrix.
Considering the upper triangular part,
\begin{equation}
	\begin{pmatrix}
		0_5&1_4&1_3&0_2&1_1&1_0
	\end{pmatrix}\label{binary_matrix}
\end{equation}
We can have the vector \ref{binary_matrix} as a number.
So converting it from binary we have $2^0+2^1+0+2^3+2^4+0$=1+2+8+16=27.
\begin{figure}[H]
	\centering
	\includegraphics[scale =0.5]{random_graphs/graph_27.png}
	\caption{Graph 27 Ord(4)}
	\label{graph 27}
\end{figure}
We can also start from a number and convert it to binary and put into the matrix.
From the adjacency matrix we can draw the graph depending on the edges, ensuring that the number of binary values is equal to the matrix upper triangular elements.This gave us a way in which we can define and label our graphs. 
And going through all the graphs for order 4, we can notice that some graphs are similar.
These graphs of the same order with different structures but same orientations are known as Homomorphic. 
A graph homomorphism f from a graph $G = (V(G),E(G))$ to a graph $H =(V(H),E(H))$
is a mapping from G to H such that each vertex in
$V(G)$ is mapped to a vertex in $V(H)$ with the same label,
and each edge in $E(G)$ is mapped to an edge in $E(H)$. \cite{fan2010graph}
\begin{figure}[H]
	\includegraphics[scale = 0.3]{homomorphic/graph_1.png}
	\includegraphics[scale = 0.3]{homomorphic/graph_2.png}
	\includegraphics[scale = 0.3]{homomorphic/graph_4.png}
	\includegraphics[scale = 0.3]{homomorphic/graph_8.png}
	\includegraphics[scale = 0.3]{homomorphic/graph_16.png}
	\includegraphics[scale = 0.3]{homomorphic/graph_32.png}
	\caption{Homomorphic Graphs Ord(4)}
	\label{Homomorphic}
\end{figure}
These graphs represent one class graphs.
Each vertex that has an edge has only one edge to another edge.
This is going to help in understanding Ramsey theory.
\newline
\subsection*{Ramsey Theory}
Ramsey Theory states given an integer $ r\geq 0$ ,every large enough graph $G = (V(G),E(G))$ contains either  $K_r $ or ${K_b}$ monochromatic clique.\cite{katz2018introduction}
If $G = (V(G), E(G))$, V(G) is a clique if it is a complete graph that is a subset $V(G)$ with order k, then $V'(G)$ is k-clique.\cite{katz2018introduction} 
Ramsey number is the minimum number of vertices R(r,b) such that all graphs of order n contain a clique of order r or an independent set of order r.
Here complete graphs are considered.\cite{burr1981generalized}
$K_n$ represents a complete graph of n vertices.
A complete graph with n vertices has ${\frac{n^2-n}{2}}$ edges.
For an order n graph, the total amount of graphs would be $2^{\left(\frac{n^2-n}{2}\right)}$.
\begin{figure}[H]
	\centering
	\includegraphics[scale = 0.5]{random_graphs/clique_2.png}
	\caption{The graph has clique $K_3$}
	\label{clique}
\end{figure}
The graph \ref{clique} has a monochromatic clique $K_3$ with blue edges.
Ramsey numbers are very hard to find because the graphs in which we try to find the monochromatic cliques grow super exponential.
Increasing with order $\sim 2^{n^2}$.
Only a few of these are known.
The highest known one is $R(4,4)= 18$ which was found by Graver and Yackel in 1968 \cite{GRAVER1968125}.
For example for order 4, we would have a total of 64 graphs and for order 5 we have 1028 graphs.
And going from this noticing that $R(3,3) = 6$ the total number of graphs of order 6 is $32768$.
And what we are currently looking to find $R(5,5)$ was found is between 43 and 46\cite{angeltveit2024r55le46}.
For $n= 43$, the total number of graphs is $2^{903}$ and 46 $2^{1035}$.
If a brute force/ check all these one by one approach is used one would have to go through all this graphs.
Which is impossible.\\
So some methods of finding these numbers are introduced, \cite{burr1981generalized} talks about generalized Ramsey Theory.
Which tries to understand other structures in Ramsey theory, not just complete graphs.
Here the computational complexity of how to find these numbers is introduced.
An asymptotic formula was introduced that explains how the numbers behave for large n.
And trying to obtain the lower and upper bounds of Ramsey numbers.
A few of these bounds have been found currently.
The paper does not address with finding certain Ramsey numbers only the bounds.
Although these are helpful, it is still very hard to find the numbers even with the bounds.\\
The second section will deal with computations and computation complexity.
Explaining why Quantum computers have more computational power over classical ones.
And how they work.	
Which would then lead to using them to find Ramsey numbers.
\section{Quantum Computation of Ramsey Numbers}
Classical computers use bits to hold information.
They can either be 1 or 0.
These usually appear in the computer as transistors.
Logic gates are machines that have bits and perform computations \cite{deutsch1985quantum}.
And Quantum computers(QC) store information in quantum bits(qubits).
These can be a superposition of both 1 and 0.
Qubits physically can be thought of as a 2-state system such as a spin-half.
Mathematically this is represented in Dirac notation as $$|1\rangle \& |0\rangle$$
As a vector these are represented as $$|0\rangle \doteq \begin{pmatrix}
	1\\0
\end{pmatrix}$$
$$|1\rangle \doteq \begin{pmatrix}
	0\\1
\end{pmatrix}$$
$|0\rangle$ and $|1\rangle$ represents states.
QC uses principles like superposition and entanglement.
This gives these computers more computational power over the classical ones.
Superposition of a state can be defined as a linear combination of these states.\cite{mcintyre_quantum_2012}
A superposition state of qubits can be expressed as \begin{equation}
|\psi\rangle = \alpha|0\rangle + \beta|1\rangle \doteq \begin{pmatrix}
	\alpha\\\beta
\end{pmatrix} \end{equation}
Where the $\alpha \& \beta$ are complex constants.
And their square of magnitude represents the probability of measuring the associated state, $P_{0}=|\alpha|^2$ and $P_{1}=|\beta|^2$. 
If n particles are added we have $2^n$ states or $2^n$ pieces of information of information. 
For n = 4,
\begin{equation}
	|\psi\rangle =c_1 |00\rangle+c_2 |01\rangle+c_3 |10\rangle+ c_4 |11\rangle
\end{equation}
Representing a new state basis. 
Where in this new state \begin{equation}|00\rangle = |0\rangle\otimes|0\rangle =|0\rangle\langle 0| \doteq
	\begin{pmatrix}
		1\\0
	\end{pmatrix} \begin{pmatrix}
		1&0
\end{pmatrix}\end{equation}
Which returns the new vectors $|00\rangle =	\begin{pmatrix}
	1\\0\\0\\0
\end{pmatrix}$ , $|01\rangle =	\begin{pmatrix}
0\\1\\0\\0
\end{pmatrix}$ ,$|10\rangle =	\begin{pmatrix}
0\\0\\1\\0
\end{pmatrix}$
$|11\rangle =	\begin{pmatrix}
	0\\0\\0\\1
\end{pmatrix}$
Superposition states is the powerful of quantum information processing because the 
amount of information contained in a quantum system grows exponentially with the n  qubits in a system.
And entanglement happens when two or more particles connect to each other.
In QC this introduces \textit{quantum parallelism} where multiple operations are performed at once.
Measuring the state of one particle determines the other's state \cite{mcintyre_quantum_2012}.
A two particle system of entangled states is mathematically expressed as \begin{equation}|\beta_{00} \rangle = a|00\rangle + b|11\rangle\end{equation}
,this is one of the Bell state of a 2-system state.
Bell states are an alternate basis to the couple and uncoupled bases\cite{mcintyre_quantum_2012}.\\
Quantum computers use quantum gates to do their computations\cite{AndrewSteane}.
Quantum gates are used to change the coefficients in qubits without destroying decoherence.
This is because when measured quantum states are sensitive once measured they remain in that state(collapse).
Quantum and logic gates use matrix multiplication to represent their transformations.
For quantum gates Pauli matrices are examples of these.
And these matrices are known as operators which are linear, meaning they act on these states \cite{AndrewSteane,mcintyre_quantum_2012}.
\begin{equation}
\sigma_x =\begin{pmatrix}
	0&1\\1&0
	\end{pmatrix},
	\sigma_y =\begin{pmatrix}
		0&-i\\i&0
	\end{pmatrix},\sigma_z =\begin{pmatrix}
	1&0\\0&-1
	\end{pmatrix}
\end{equation}
Are the Pauli matrices these transform the initial state $\pi$ around their respective basis.
And \begin{equation}H=\frac{1}{\sqrt{2}}
	\begin{pmatrix}
		1&1\\1&-1
	\end{pmatrix}
\end{equation} is the Hadamard gate, this is important because it helps to turn a state into a superposition of the 0 and 1 states.
\\
\\
This article \cite{Deutsch1989} explains quantum computing by expanding on classical computation.
What the challenge at the time was there was lack of quantum computational models.
With this, Deutsch tries how to figure build quantum networks that can do computations like logic gates.
Using the classical as a reference,these quantum gates have to include quantum mechanics principles like superposition.
These quantum network takes qubits as inputs, gates for computation and measurements as outputs.
Deutsch gives a picture of what a quantum machine does and capable of.
And shows how in theory Quantum computers have more computational power over classical ones.
Although these were a great theoretical model, no physical implementations are included.
Even the way in which quantum computers work is not included in the paper.
And also it does not show how this network would break only shows theory.
\\\\
Here in the paper \cite{doi:10.1137/S0097539796298637}, Simon tries to see if it would be possible for quantum computer to be faster than classical computers.
What was needed to be found is a problem that can prove this.
Something called the Oracle task is defined.
This tasks wants to see if a function is two-to-one and if so find a hidden string.
For a classical computer for this problem one has to go through $ 2^{\left(\frac{n}{2}\right)}$ times of the function to get a solution. 
Then Simon provided a way of using quantum computers to solve this.
Using quantum circuit from \cite{Deutsch1989} to go through the function in superposition.
Quantum superposition checks all inputs at once and interference is used to see the correct output.
This showed that this quantum algorithm solves this faster than the classical.
\newpage
\cite{rietsche2022quantum} gives an overview of quantum computing.
Seeing that many fields might need quantum computers to help solve problems, a knowledge of it is needed.
These organizations don’t know how to leverage quantum computing practically.
Here this article addresses this by providing three ways in which a quantum computer operates.
Quantum annealing is used an analog of how they operate.
Also by providing different applications that can be used like optimization.
Search and graph is used as one of examples of optimization applications.
After defining these, another way to deal with this was proposed is to introduce research areas in different places.
Different way of looking at this paper is that it assumes breakthroughs in error correction.
\newpage
\section*{Comments}


From reading the paper \cite{burr1981generalized}, here I see they speak on the computational complexity of Ramsey Numbers. 
I have a little trouble in understanding Ramsey numbers as a class of $\sum_{2}^{p}$. 
I also do not get the notation or how Ramsey numbers are part of this class.
The mathematical notation in this paper is the most confusing, like what $W_{1,n}$ n-spoked wheel represent.
\\
\\
\cite{Deutsch1989}
A complete set C of compatible observables for $\mathcal{P}$ is chosen, including an input 
observable $\hat{I}$ and an observable $\hat{h}$ that is independent of $\hat{I}$ and has spectrum \{0,1\}, the halt flag. 
This part in the paper is confusing.
A halt flag is introduced not sure of its importance but im confused by this.
I do not understand how from this set what compatible observables are chosen.
\\
\\
I notice I am very bad at getting the main ideas from papers.
And even the main aims of the papers.
\\
\\
In this paper \cite{doi:10.1137/S0097539796298637}, asks 'if a function is invariant under XOR-mask' I do not understand this part.
Not fully sure what an "XOR-mask"
Majority of this paper speaks about "Oracle".
I feel like the reason i dont understand these parts is because the paper is more computer science based/ the notation of computer science is used and my background is poor. 
\section*{Previous draft comments}
\newpage
\bibliography{sample.bib}
\end{document}

.\newline
\subsection{Quantum Networks}

A logic gate for quantum computation can be represented as a matrix $S$.
For computation to be reversible $S$ has to be unitary or $$S^*S=SS^* =I$$
Depending on the number of particles n, $S$ would have dimensions $2^n\times2^n$.
An inverse of an $S$ matrix is Hermitian.

Here we are introduced to quantum computational networks.
Quantum computations that use quantum coherence are reversible.








