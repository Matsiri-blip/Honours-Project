\documentclass{Assignment}
% ------------------------------------------------------------


% ------------------------------------------------------------
% Formatting
% ------------------------------------------------------------
\usepackage{color}
\usepackage{fullpage}
% ------------------------------------------------------------


% ------------------------------------------------------------
% Bibliography
% ------------------------------------------------------------
\usepackage{doi}
\usepackage{hyperref}
\usepackage{usnomencl}
\usepackage[square,sort&compress,comma,numbers]{natbib}
\bibliographystyle{unsrtnat}
\hypersetup{
	colorlinks,
	citecolor=blue,
	linkcolor=red,
	urlcolor=red}
\usepackage{bibentry}
\nobibliography*
% ------------------------------------------------------------


\author{Matsiri Madiba}
\date{March 2025}
\title{Annotated Bibliography}

\begin{document}
\maketitle
\cite{burr1981generalized}~\bibentry{burr1981generalized}
\begin{description}
	\item[Aim:]
	To investigate generalized Ramsey theory, which extends on classical Ramsey theory.
	Find important ideas found in Ramsey theory for graphs while also focusing on Ramsey numbers.

\item[Style:]
Review Paper
\item[Cross References:]

For Ramsey theory only complete graphs are considered. 
Burr tries to understand how hard evaluating Ramsey Numbers is.
By introducing different complexities of problems.
The article \cite{PhysRevA.93.032301} tries getting around this by using Quantum Computing to find these numbers.
\item [Summary:]
Ramsey theory tries to understand the patterns that occur within graphs.
Finding Ramsey numbers is difficult.
They can be defined as part of NP-complete problem or even harder.
Which are a class of difficult problems.
Generalized Ramsey Theory does not only consider complete graphs.
We have Ramsey numbers involving a fixed graph with increasing n, try to find exact results for all n. 
Then an asymptotic formula at the bounds is used to give information about these numbers.  
\end{description}

\newpage
\cite{PhysRevA.93.032301}~\bibentry{PhysRevA.93.032301}
\begin{description}
	\item [Aim:]
	To find Ramsey numbers using quantum computer algorithms.
	\item [Style:]
	Theoretical Research paper
	\item [Cross References:]
	Quantum algorithm is used to calculate Ramsey Numbers, proposed in \cite{gaitan2012ramsey}.
	A Hamiltonian is used, this was also referenced in the paper \cite{doi:10.1137/S0097539796298637}.
	Wang turned the problem into a combinatorial optimization problem.
	Adiabatic Quantum Optimization considering \cite{farhi2000quantum} was used.
	
	\item [Summary:]
	Ramsey numbers are very difficult to calculate.
	Quantum computers provide a computational speedup compared to classical ones.
	A Hamiltonian of the problem gets defined.
	If ground-state energy($E_1$) of a Hamiltonian is zero a Ramsey number exist. 
	And using bounds of Ramsey Number n is increased until the Hamiltonian $E_1$ is larger than 0.
	The qubit measures Hamiltonian energy spectrum by varying reference state energy. 
	And when the n is not a Ramsey Number the qubits exhibits a resonance dynamics.
	This means they will oscillate at a larger frequency.
\end{description}


\newpage
\cite{Deutsch1989}~\bibentry{Deutsch1989}
	    \begin{description}
			\item [ Aim:]
		To understand the theory of quantum computations using understanding of classical ones.
		By using the ideas of classical gates and extending them to quantum gates.
		\item[Style:]
		Journal article, theoretical.
	
		\item[Cross References:]
		Deutsch builds on universal quantum computer by \cite{deutsch1985quantum} from classical computing to quantum mechanics. 
		We understand about the development of quantum computation theory.
		He then relates universal quantum computer to the classical computer using the paper \cite{Turing1936}.
		\item [Summary:]
		Computation can be thought as a mathematical function that has an input, and gets an output from that.
		Theory of computations deals with the complexity of a problem.
		Quantum computations could be represented as networks of quantum gates.
		A universal quantum gate simplifies quantum computing theory.
		It makes building quantum computers more practical. 
		Quantum coherence gives them computational power over classical ones.
	\end{description}

	\newpage

\cite{doi:10.1137/S0097539796298637}~\bibentry{doi:10.1137/S0097539796298637}
\begin{description}
	\item[Aim:]
	To present a quantum algorithm that is fast.
	\item [Style:]
	Theoretical Research Paper
	\item [Cross References:]
	This article speaks about quantum machines from the paper \cite{Deutsch1989}.
	Simon gives reasons why quantum computers have advantage over classical ones.
	The focus is on quantum computing algorithm that can show how fast these computers are	.
	Using the article \cite{shor1994algorithms} he explores the development of quantum polynomial-time algorithms.
	\item[Summary:]
	 In theory quantum computers can solve certain computational problems fast.
	They use interference to increase or decrease the probability of a given state.
	There is a quantum algorithm that can distinguish between two classes of polynomial-time computable functions in polynomial time. 
	This suggests a quantum advantage in function classification. 
\end{description}

\newpage
\cite{UQS}~\bibentry{UQS}

\begin{description}
	\item[Aim:] 
	To show that quantum computers can simulate nature better.
	
	\item [Style:] Theoretical Research Paper
	
	\item [Cross References:]
	Lloyd introduces the concept of universal quantum simulators. 
	Which extends on the idea introduced by Feynman of a computer that can simulate quantum systems from the paper \cite{feynman1982simulating}. 
	He also speaks on quantum gates, which was also referenced in the paper \cite{Deutsch1989}
	
	\item[Summary:] 
	Quantum computers are able to simulate quantum systems. 
	Showing the potential for quantum computers to simulate many physical processes.	
	Showing how quantum computers offer an exponential speedup.

	
	
\end{description}
\newpage
\cite{AndrewSteane}~\bibentry{AndrewSteane}
\begin{description}
	\item[Aim:] 
	To have a basic understanding of quantum computers strengths and weaknesses
	
	\item [Style:] 
	Review paper
	
	\item [Cross References:] 
	Here Steane speaks about the physical implementations of quantum computers. 
	The concept of quantum computing developed in the early 1980s from papers \cite{feynman1982simulating}. 	
	In the papers \cite{PhysRevA.93.032301}, \cite{doi:10.1137/S0097539796298637} and \cite{UQS} developments and applications are discussed.
	
	\item[Summary:] 
	Qubits represent the basic unit of quantum information.
	They can be entangled.
	Qubits can be manipulated using quantum gates.
	Due to sensitivity of quantum computers to the external environment, Quantum Error Correlation was introduced.
	Quantum error correction (QEC) ensures that a quantum computation remains  accurate.
\end{description}
	 	\newpage
	 	\bibliography{sample}
\end{document}


