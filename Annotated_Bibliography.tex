\documentclass{Assignment}
% ------------------------------------------------------------


% ------------------------------------------------------------
% Formatting
% ------------------------------------------------------------
\usepackage{color}
\usepackage{fullpage}
% ------------------------------------------------------------


% ------------------------------------------------------------
% Bibliography
% ------------------------------------------------------------
\usepackage{doi}
\usepackage{hyperref}
\usepackage{usnomencl}
\usepackage[square,sort&compress,comma,numbers]{natbib}
\bibliographystyle{unsrtnat}
\hypersetup{
	colorlinks,
	citecolor=red,
	linkcolor=red,
	urlcolor=red}
\usepackage{bibentry}
\nobibliography*
% ------------------------------------------------------------


\author{Matsiri Madiba}
\date{March 2025}
\title{Ramsey Numbers and Quantum Computing}

\begin{document}
\maketitle
\cite{burr1981generalized}~\bibentry{burr1981generalized}
\begin{description}
	\item[Aim:]
To find the guiding ideas that have contributed to the discoveries in Ramsey theory for graphs and finding Ramsey numbers.
\item[Style:]
Review Paper
\item[Cross References:]
We see that it extends on a 15 year old research at the time(1981). They only consider complete graphs. We see Ramsey theory explored by also considering the book \cite{graham1980ramsey}, while also exploring complex graph structures.
\item [Summary:]
Ramsey theory deals with graphs, and tries to understand the patterns that occur in those graphs.
Finding Ramsey numbers is not very easy.
They can be defined as part of NP-complete problem.
We see how an NP-complete problems are important.
Here we see how to determine Ramsey numbers.  
We have Ramsey numbers involving a fixed graph with growing graphs of size n, to find them aiming for exact results for all n. 
Then an asymptotic formula at minimum, upper and lower bounds is used to estimate these numbers.  

\end{description}
\newpage
\cite{f3197c3a-bce3-3c72-b6d6-c1ee34ac9328}~\bibentry{f3197c3a-bce3-3c72-b6d6-c1ee34ac9328}
	    \begin{description}
			\item [ Aim:]
		To create an understanding of quantum computation that would lead to the practical usage and creation of quantum algorithms.
		\item[Style:]
		Journal article, theoretical.
		\item[Cross References:]
		It extends more on universal quantum computer by \cite{deutsch1985quantum}, which can simulate a quantum system. 
		We understand about developing the theoretical basis for quantum computation, not on summarizing existing research. 
		It extends a bit on the theory between known classical computations. 
		Together with few known quantum mechanics algorithms that have existed for few years prior to the release of the paper.
		\item [Summary:]
		Deutsch talks about quantum basics like compatible observables, superposition, entanglement and the Hamiltonian.
		He emphasized that quantum computations could be represented as networks of quantum gates, similar to classical circuits.
		Quantum logic and gates transform quantum binary inputs into quantum binary outputs regardless of the input values. To avoid repetition since these gates have same definitions.
		There are various gates and each of them have different types of functions.
		Then the paper explained universal quantum gate
		So a universal quantum gate simplifies quantum computing theory and makes building quantum computers more practical. It then shows the importance of a Hamiltonian in computers.
	\end{description}

	
	 	\newpage
	 	\bibliography{sample}
\end{document}


