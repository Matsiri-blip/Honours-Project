\documentclass{Assignment}
% ------------------------------------------------------------


% ------------------------------------------------------------
% Formatting
% ------------------------------------------------------------
\usepackage{color}
\usepackage{fullpage}
% ------------------------------------------------------------


% ------------------------------------------------------------
% Bibliography
% ------------------------------------------------------------
\usepackage{doi}
\usepackage{hyperref}
\usepackage{usnomencl}
\usepackage[square,sort&compress,comma,numbers]{natbib}
\bibliographystyle{unsrtnat}
\hypersetup{
	colorlinks,
	citecolor=red,
	linkcolor=red,
	urlcolor=red}
\usepackage{bibentry}
\nobibliography*
% ------------------------------------------------------------


\author{Matsiri Madiba}
\date{March 2025}
\title{Ramsey Numbers and Quantum Computing}

\begin{document}
\maketitle
\cite{burr1981generalized}~\bibentry{burr1981generalized}
\begin{description}
	\item[Aim:]
To find the guiding ideas that have contributed to the discoveries in Ramsey theory for graphs and finding Ramsey numbers.
\item[Style:]
Review Paper
\item[Cross References:]
We see that it extends on a 15 year old research at the time(1981). They only consider complete graphs. We see Ramsey theory explored by also considering the book \cite{graham1980ramsey}, while also exploring complex graph structures.
\item [Summary:]
Ramsey theory deals with graphs, and tries to understand the patterns that occur in those graphs.
Finding Ramsey numbers is not very easy.
They can be defined as part of NP-complete problem.
We see how an NP-complete problems are important.
Here we see how to determine Ramsey numbers.  
We have Ramsey numbers involving a fixed graph with growing graphs of size n, to find them aiming for exact results for all n. 
Then an asymptotic formula at minimum, upper and lower bounds is used to estimate these numbers.  

\end{description}

	 	\newpage
	 	\bibliography{sample}
\end{document}


