\documentclass{Assignment}
% ------------------------------------------------------------


% ------------------------------------------------------------
% Formatting
% ------------------------------------------------------------
\usepackage{color}
\usepackage{fullpage}
% ------------------------------------------------------------


% ------------------------------------------------------------
% Bibliography
% ------------------------------------------------------------
\usepackage{doi}
\usepackage{hyperref}
\usepackage{usnomencl}
\usepackage[square,sort&compress,comma,numbers]{natbib}
\bibliographystyle{unsrtnat}
\hypersetup{
	colorlinks,
	citecolor=blue,
	linkcolor=red,
	urlcolor=red}
\usepackage{bibentry}
\nobibliography*
% ------------------------------------------------------------


\author{Matsiri Madiba}
\date{March 2025}
\title{Annotated Bibliography}

\begin{document}
\maketitle
\cite{burr1981generalized}~\bibentry{burr1981generalized}
\begin{description}
	\item[Aim:]
To get the ideas that have helped to the make findings in Ramsey theory for graphs.
\item[Style:]
Review Paper
\item[Cross References:]
Ramsey theory is explored by considering the book \cite{graham1980ramsey}.
For Ramsey theory complete graphs are considered. 
Burr tries to understand how hard evaluating Ramsey Numbers is.
We are introduced to different complexity of the problem.
The paper \cite{PhysRevA.93.032301} they try getting around this by using Quantum Computing.
And in this paper different graph structures are explored not just complete graphs.
\item [Summary:]
Ramsey theory tries to understand the patterns that occur within graphs.
Finding Ramsey numbers is difficult.
They can be defined as part of NP-complete problem or even harder.
Which are a class of difficult problems.
Generalized Ramsey Theory does not only consider complete graphs.
Complex graph structures are considered .
We have Ramsey numbers involving a fixed graph with growing graphs of size n to find exact results for all n. 
Then an asymptotic formula at the bounds is used to give information about these numbers.  
\end{description}

\newpage
\cite{PhysRevA.93.032301}~\bibentry{PhysRevA.93.032301}
\begin{description}
	\item [Aim:]
	To find Ramsey numbers using a quantum computer algorithms.
	\item [Style:]
	Theoretical Research paper
	\item [Cross References:]
	Quantum algorithm was used to calculate Ramsey Numbers which was proposed by the paper \cite{gaitan2012ramsey}.
	Some Ramsey theory is explored like in the paper \cite{burr1981generalized} and book \cite{graham1980ramsey}.
	Wang turned Ramsey numbers into a decision problem on a quantum computer.
	A humiliation is used, this was also introduced in the paper \cite{Deutsch1989}.
	This was done by turning the problem into a combinatorial optimization problem.
	He used Adiabatic Quantum Optimization from \cite{farhi2000quantum}.
	
	\item [Summary]
	Ramsey numbers are very difficult to calculate.
	Wang proposes to use a quantum algorithm.
	This provides a computational speedup compared to classical methods.
	If ground-state energy($E_1$) of a Hamiltonian is zero a Ramsey number exist. 
	And using bounds of Ramsey Number n is increased until the Hamiltonian $E_1$ is larger than 0.
	The algorithm uses concepts like probe qubits, adiabatic quantum evolution and resonance dynamics. 
	The probe qubit measures Hamiltonian energy spectrum by varying reference state energy. 
	And when the n is not a Ramsey Number we have that the probe qubits exhibits a resonance dynamics.
	This means they will oscillate at a larger frequency.
\end{description}


\newpage
\cite{Deutsch1989}~\bibentry{Deutsch1989}
	    \begin{description}
			\item [ Aim:]
		To understand quantum computation and creation of quantum algorithms.
		\item[Style:]
		Journal article, theoretical.
	
		\item[Cross References:]
		Deutsch extends more on universal quantum computer by \cite{deutsch1985quantum} from classical computing to quantum mechanics. 
		We understand about the development of quantum computation theory.
		He also includes the theory between known classical computations. 
		He then relates universal quantum computer to the classical computer using the paper \cite{Turing1936}.
		\item [Summary:]
		Computation can be thought as a mathematical function that has an input, and gets an output from that.
		Theory of computations deals with the complexity of a problem.
		Quantum computations could be represented as networks of quantum gates.
		Quantum gates differ from logic gates in where the inputs and outputs of quantum gates can be a mixture of states.
		Quantum logic gates transform quantum binary inputs into quantum binary outputs independent of to the input values.
		A universal quantum gate simplifies quantum computing theory, and makes building quantum computers more practical. 
		Quantum coherence gives them computational power over classical ones.
	\end{description}

	\newpage

\cite{doi:10.1137/S0097539796298637}~\bibentry{doi:10.1137/S0097539796298637}
\begin{description}
	\item[Aim:]
	To present a quantum algorithm that is fast.
	\item [Style:]
	Theoretical Research Paper
	\item [Cross References:]
	This article speaks about quantum machines from the paper \cite{Deutsch1989}.
	Simon gives reasons why quantum computers have advantage over classical ones.
	The focus is on quantum computing algorithm that can show how fast these computers are	.
	Using the paper \cite{shor1994algorithms} he explores the development of quantum polynomial-time algorithms.
	\item[Summary:]
	 In theory quantum computers can solve certain computational problems fast.
	They use interference to increase or decrease the probability of a given state.
	There is a quantum algorithm that can distinguish between two classes of polynomial-time computable functions in polynomial time. 
	This suggests a quantum advantage in function classification. 
\end{description}

\newpage
\cite{UQS}~\bibentry{UQS}

\begin{description}
	\item[Aim:] 
	To show that quantum computers can simulate nature better.
	
	\item [Style:] Theoretical Research Paper
	
	\item [Cross References:]
	Lloyd introduces the concept of universal quantum simulators. 
	Which extends on the idea introduced by Feynman of a computer that can simulate quantum systems from the paper \cite{feynman1982simulating}. 
	Using the paper \cite{Deutsch1989} he also speaks on quantum gates
	
	\item[Summary:] 
	Quantum computers are able to simulate quantum systems. 
	Showing the potential for quantum computers to simulate many physical processes.	
	Showing how quantum computers offer an exponential speedup.

	
	
\end{description}
\newpage
\cite{AndrewSteane}~\bibentry{AndrewSteane}
\begin{description}
	\item[Aim:] 
	To have a basic understanding of quantum computers strengths and weaknesses
	
	\item [Style:] 
	Review paper
	
	\item [Cross References:] 
	Here Steane speaks about the physical implementations of quantum computers. 
	The concept of quantum computing developed in the early 1980s from papers \cite{feynman1982simulating}. 	
	In the papers \cite{PhysRevA.93.032301}, \cite{doi:10.1137/S0097539796298637} and \cite{UQS} developments and applications are discussed.
	
	\item[Summary:] 
	Qubits represent the basic unit of quantum information.
	They can be entangled, a phenomenon where two or more qubits become connected, even when separated by large distances.
	Qubits can be manipulated using quantum gates.
	Due to sensitivity of quantum computers to the external environment, Quantum Error Correlation was introduced.
	Quantum error correction (QEC) ensures that a quantum computation remains  accurate.
\end{description}
	 	\newpage
	 	\bibliography{sample}
\end{document}


